\chapter{Análisis de la Solución}

En el presente capítulo se describirá el proceso de análisis de la solución, para ello se debe considerar que existe una propuesta de solución, la cual fue descrita en el capítulo anterior, y que será la base para el siguiente análisis. Para llevar a cabo este análisis se debe tomar en cuenta todas aquellas características con las que debe contar el sistema, ya que estas serán la base de la cual partir. 

\section{Análisis de requerimientos}
A continuación se describirán los requerimientos \hyperlink{cv:funcionales}{funcionales} y \hyperlink{cv:noFuncionales}{no funcionales} con los que debe contar el sistema para su correcto desarrollo.

\hypertarget{cv:funcionales}{\subsection{Funcionales}}

El sistema deberá:

\begin{itemize}
	\item Permitir el registro de una zona turística.
	\item Permitir el trazado de un área geográfica de interés para la zona turística.
	\item Notificar a un turista que ha ingresado a una zona turística previamente registrada y delimitada.
	\item Mostrar la información de la zona turística al turista.
	\item Mostrar los servicios turísticos de la zona al turista.
	\item Crear rutas turísticas.
\end{itemize}

\hypertarget{cv:noFuncionales}{\subsection{No funcionales}}

El sistema deberá tener: 

\begin{itemize}
	\item Alta disponibilidad.
	\item Geolocalización.
	\item Alta usabilidad
\end{itemize}