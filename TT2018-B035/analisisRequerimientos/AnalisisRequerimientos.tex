\hypertarget{cv:analisisSol
	ucion}{
	\chapter{Análisis de la Solución}
}

En el presente capítulo se describirá el proceso de análisis de la solución, para ello se debe considerar que existe una propuesta de solución, la cual fue descrita en el capítulo anterior, y que será la base para el siguiente análisis. Para llevar a cabo este análisis se debe tomar en cuenta todas aquellas características con las que debe contar el sistema, ya que estas serán la base de la cual partir. \\ 

Dentro del análisis de la solución se contemplará: \textbf{análisis de requerimientos}, \textbf{actores del sistema}, \textbf{modelo de datos}, \textbf{diagrama de clases}, \textbf{diagrama de secuencia} y \textbf{análisis de casos de uso}. Dichos procesos serán descritos en las siguientes secciones.

\section{Análisis de requerimientos}
A continuación se describirán los requerimientos \hyperlink{cv:funcionales}{funcionales} y \hyperlink{cv:noFuncionales}{no funcionales} con los que debe contar el sistema para su correcto desarrollo. Para la identificación de los requerimientos se utilizará la siguiente nomenclatura: 

\begin{center}
\Huge{RNF / RF \textit{n}: \textit{descripción\_requerimiento}}
\end{center} 

Donde: 

\begin{itemize}
	\item RNF: significa Requerimiento No Funcional
	\item RF: significa Requerimiento Funcional
	\item \textit{n}: significa número de requerimiento
	\item \textit{descripción\_requerimiento}: significa una breve descripción del requerimiento
\end{itemize}

Una vez determinada la nomenclatura para identificar a los requerimientos pasaremos a describirlos: 
\hypertarget{cv:funcionales}{\subsection{Funcionales}}

El sistema deberá:

\begin{itemize}
	\item RF1: Permitir el registro de una zona turística.
	\item RF2: Permitir el trazado de un área geográfica de interés para la zona turística.
	\item RF3: Notificar a un turista que ha ingresado a una zona turística previamente registrada y delimitada.
	\item RF4: Mostrar la información de la zona turística al turista.
	\item RF5: Mostrar los servicios turísticos de la zona al turista.
	\item RF6: Crear rutas turísticas.
\end{itemize}

\hypertarget{cv:noFuncionales}{\subsection{No funcionales}}

El sistema deberá tener: 

\begin{itemize}
	\item RNF1: Alta disponibilidad.
	\item RNF2: Geolocalización.
	\item RNF3: Alta usabilidad
\end{itemize}

\section{Actores del sistema}

\begin{Usuario}{\hypertarget{actor:turista}{\subsection{Turista}}}{
		Es el usuario final del sistema. Será el encargado de consumir la información proporcionada por el sistema para la generación de rutas turísticas, así mismo éste podrá registrar comentarios a los diferentes servicios turísticos que visite.
	}
	\item[Responsabilidades:] \cdtEmpty
		\begin{itemize}
			\item Visualizar la información de las zonas turísticas.
			\item Visualizar la información de los servicios turísticos.
			\item Generar rutas turísticas.
			\item Registrar comentarios a los servicios turísticos que visite.
		\end{itemize}
	\item[Perfil:] \cdtEmpty
		\begin{itemize}
			\item Ser usuario de iPhone\footnote{Ver \url{https://www.apple.com/mx/iphone/}}.
			\item Contar con Apple ID\footnote{Ver \url{https://appleid.apple.com/\#!\&page=signin}}.
		\end{itemize}
\end{Usuario}

\begin{Usuario}{\hypertarget{actor:administrador}{\subsection{Administrador}}}{
		Es el encargado de gestionar la información de las zonas turísticas que se encontrarán dentro del sistema, esto incluye zonas turísticas y servicios turísticos.
	}
	\item[Responsabilidades:] \cdtEmpty
		\begin{itemize}
			\item Gestionar las zonas turísticas dentro del sistema.
			\item Gestionar los servicios turísticos que pertenecen a las zonas turísticas del sistema.
		\end{itemize}
	\item[Perfil:] \cdtEmpty
		\begin{itemize}
			\item Amplio conocimiento en el sector turístico.
		\end{itemize}
\end{Usuario}

\section{Modelo de Datos}
Para el modelo de datos se llevó a cabo un análisis de los datos que deberán ser almacenados en la base de datos del proyecto. Para tener una idea más clara de los datos que serán empleados se realizó, mediante un diagrama UML de modelo relacional, una propuesta para la base de datos. La Figura \ref{fig:base_datos_v1} muestra el diagrama.

\hypertarget{fig:base_datos_v1}{
	\begin{figure}[htbp]
		\begin{center}
			\hypertarget{fig:base_datos_v1}{
				\includegraphics[angle=90, scale=.4]{casosDeUso/images/base_datos_v1.jpg}
				\caption{Base de datos}
			}
			\label{fig:base_datos_v1}
		\end{center}
	\end{figure}
}

\newpage
\section{Diagrama de clases}
Un diagrama de clases sirve para representar gráficamente la estructura de un sistema cuya implementación se llevará a cabo mediante un lenguaje de programación orientado a objetos. Dentro de la etapa de diseño y posterior al análisis de requerimientos es que se realiza el diagrama de clases para el sistema. El principal objetivo de estos diagramas es representar las clases y su contenido, así como su relación con otras clases \cite{clases}.

\subsection{Componentes de un diagrama de clases}
A continuación se describirán los componentes de un diagrama de clases, estos componentes se encuentran dentro del Lenguaje Unificado de Modelado (Unified Modeling Language, UML\footnote{\url{https://www.uml.org/}} por sus siglas en ingles)\footnote{De aquí en adelante se empleará UML para referirse al Lenguaje Unificado de Modelado}.

\begin{itemize}
	\item \textbf{Clase}: Es el componente básico para el diagrma de clases, estas representan las entidades o conceptos. Dentro de las clases se definen los atributos y métodos que utilizarán los objetos de la clase. La Figura \ref{fig:clase} muestra un ejemplo de como se representa una clase.
	
	\hypertarget{fig:clase}{
		\begin{figure}[htbp]
			\begin{center}
				\hypertarget{fig:clase}{
					\includegraphics[ scale=.8]{analisisRequerimientos/clases/images/clase}
					\caption{Ejemplo de clase}
				}
				\label{fig:clase}
			\end{center}
		\end{figure}
	}

	\item \textbf{Atributos y métodos}: Los atributos generalmente se muestran sus nombres y con su tipo, mientras que los métodos además de su nombre se incluye su tipo de retorno, en caso de que tenga, y los parámetros de entrada que recibe. Así mismo, tanto atributos como métodos están acompañados de un símbolo antes de su nombre, estos símbolos representan:
	
	\begin{itemize}
		\item \textbf{+} representa atributos públicos.
		\item \textbf{-} representa atributos privados.
		\item \textbf{\#} representa atributos protegidos.
	\end{itemize}

	\item \textbf{Relaciones}: Ya que las clases se relacionan entre sí, existen distintos tipos de relaciones:
	
	\begin{itemize}
		\item \textbf{Generalización}: Representa una extensión o herencia de una clase de otra. La Figura \ref{fig:generalizacion} muestra un ejemplo de esta relación.
		
		\hypertarget{fig:generalizacion}{
			\begin{figure}[htbp]
				\begin{center}
					\hypertarget{fig:generalizacion}{
						\includegraphics[ scale=.8]{analisisRequerimientos/clases/images/generalizacion}
						\caption{Ejemplo de generalizacion}
					}
					\label{fig:generalizacion}
				\end{center}
			\end{figure}
		}
		
		\newpage
		\item \textbf{Asociación}: Es una relación básica entre dos clases. Pueden ser unidireccionales (Figura \ref{fig:unidireccional}) o bidireccional (Figura \ref{fig:bidireccional}). 
		
		\hypertarget{fig:unidireccional}{
			\begin{figure}[htbp]
				\begin{center}
					\hypertarget{fig:unidireccional}{
						\includegraphics[ scale=.8]{analisisRequerimientos/clases/images/unidireccional}
						\caption{Ejemplo de unidireccional}
					}
					\label{fig:unidireccional}
				\end{center}
			\end{figure}
		}
		
		\hypertarget{fig:bidireccional}{
			\begin{figure}[htbp]
				\begin{center}
					\hypertarget{fig:bidireccional}{
						\includegraphics[ scale=.8]{analisisRequerimientos/clases/images/bidireccional}
						\caption{Ejemplo de bidireccional}
					}
					\label{fig:bidireccional}
				\end{center}
			\end{figure}
		}
	
		\item \textbf{Agregación}: Representa que un objeto de una clase contiene objetos de otra clase. La Figura \ref{fig:agragacion} muestra un ejemplo de una agregación.
		
		\hypertarget{fig:agragacion}{
			\begin{figure}[htbp]
				\begin{center}
					\hypertarget{fig:agragacion}{
						\includegraphics[ scale=.8]{analisisRequerimientos/clases/images/agregacion}
						\caption{Ejemplo de agragacion}
					}
					\label{fig:agragacion}
				\end{center}
			\end{figure}
		}
		
		\newpage
		\item \textbf{Composición}: Es una agregación fuerte, es decir, si un objeto contenido dentro de otra clase deja de existir no tiene sentido que el objeto contenedor siga existiendo. La Figura \ref{fig:composicion} muestra un ejemplo de una compisición.
		
			\hypertarget{fig:composicion}{
			\begin{figure}[htbp]
				\begin{center}
					\hypertarget{fig:composicion}{
						\includegraphics[ scale=.8]{analisisRequerimientos/clases/images/composicion}
						\caption{Ejemplo de composicion}
					}
					\label{fig:composicion}
				\end{center}
			\end{figure}
		}
		
	\end{itemize}
\end{itemize}

\subsection{Diagrama de clases del sistema}
La Figura \ref{fig:diagramaClases} muestra el diagrama de clases que fue definido para el desarrollo del proyecto.

\hypertarget{fig:diagramaClases}{
	\begin{figure}[htbp]
		\begin{center}
			\hypertarget{fig:diagramaClases}{
				\includegraphics[ scale=.4]{analisisRequerimientos/clases/images/diagramaClases}
				\caption{Diagrama de clases del sistema}
			}
			\label{fig:diagramaClases}
		\end{center}
	\end{figure}
}
\newpage
\section{Diagramas de Secuencia}
El diagrama de secuencia es el encargado de mostrar la interacción que tiene un conjunto de objetos pertenecientes a una apliación en un cierto lapso de tiempo, en el cual se indican los módulos o clases que forman parte del sistema y las llamadas entre ellos para realizar una tarea específica. Estos diagramas permiten observar la perspectiva cronológica de las interacciónes que tiene el sistema \cite{secuencia}

\subsection{Diagrama de secuencia del sistema}
La Figura \ref{fig:diagramaSecuencia} muestra el diagrama de secuencia del sistema.

\hypertarget{fig:diagramaSecuencia}{
	\begin{figure}[htbp]
		\begin{center}
			\hypertarget{fig:diagramaSecuencia}{
				\includegraphics[ scale=.4]{analisisRequerimientos/secuencia/images/diagramaSecuencia}
				\caption{Diagrama de secuencia del sistema}
			}
			\label{fig:diagramaSecuencia}
		\end{center}
	\end{figure}
}

\newpage
\section{Análisis de casos de uso}
Para el análisis de casos de uso se tomaron en cuenta los requerimientos funcionales y los no funcionales, descritos en la sección anterior, que fueron identificados a partir de la propuesta de solución. \\

Para poder llevar a cabo un correcto análisis de casos de uso primero se tuvo que tener bien identificados a loa actores o usuarios finales que participarán en el uso del sistema. Una vez identificados se realizó una arquitectura de solución en la cual se definieron los módulos a desarrollar para el sistema, el análisis de casos de uso parte tomando como base estos módulos ya que en ellos será donde se centre la interacción del usuario con el sistema. \\

En la Figura \ref{fig:casosDeUso} se muestra el diagrama general de los casos de uso identificados para el sistema, en él puede observarse que los casos de uso se encuentran divididos por paquetes que corresponden a las aplicaciones que conforman el sistema, y que cada uno de estos tiene un nombre.


\hypertarget{fig:casosDeUso}{
	\begin{figure}[htbp]
		\begin{center}
			\hypertarget{fig:casosDeUso}{
				\includegraphics[angle=90, scale=.4]{casosDeUso/images/DiagramaDeCasosDeUso}
				\caption{Diagrama de casos de uso}
			}
			\label{fig:casosDeUso}
		\end{center}
	\end{figure}
}

\newpage
A continuación se muestra la nomenclatura utilizada para asignar el nombre a los casos de uso:

\begin{center}
	\Huge{\textit{Iniciales\_módulo}-CU\textit{n}: \textit{nombre\_caso\_de\_uso}}
\end{center}

Donde: 

\begin{itemize}
	\item \textbf{iniciales\_módulo}: significa que son las iniciales del módulo en el que se encuentra el caso de uso, los módulos son: 
	\begin{itemize}
		\item RE: Registro
		\item LR: Localización de rutas
		\item PO: Posicionamiento
		\item II: Interacción de la información
		\item SE: Servicios
		\item IR: Información y representación de estadísticas turísticas
		\item RA: Registro de área turística
	\end{itemize}

	\item \textbf{n}: es el número de caso de uso.
	
	\item \textbf{nombre\_caso\_de\_uso}: es el nombre del caso de uso.
\end{itemize}

Para realizar la documentación se tomaron los casos de uso y se separaron por módulo. Cabe mencionar que aunque sean distintos módulos los casos de uso tienen una relación entre sí para poder construir el sistema. \\

La documentación de los casos de uso se divide en tres partes:

\begin{enumerate}
	\item Descripción
	\item Atributos
	\item Trayectorias
\end{enumerate}

La descripción es un breve resumen de lo que trata el caso de uso, en ella se describe el por qué del caso de uso, para qué es importante tenerlo, el objetivo principal y lo que se obtiene al ejecutarlo.\\

Los atributos son descritos en una tabla, esto facilita al desarrollador identificar las entradas y salidas del caso de uso, su precedencia y los cambios que produce al sistema. \\

La trayectoria son los pasos a seguir para la correcta ejecución y desarrollo del caso de uso, en esta puede haber trayectorias alternativas, puntos de extensión a otros casos de uso o inclusión de otros casos casos de uso. \\

A continuación se detallan los casos de uso:

% \IUref{IUAdmPS}{Administrar Planta de Selección}
% \IUref{IUModPS}{Modificar Planta de Selección}
% \IUref{IUEliPS}{Eliminar Planta de Selección}

% 


% Copie este bloque por cada caso de uso:
%-------------------------------------- COMIENZA descripción del caso de uso.

%\begin{UseCase}[archivo de imágen]{UCX}{Nombre del Caso de uso}{
%--------------------------------------
	\begin{UseCase}{SE-CU2.1.2.1.1}{Registrar Comentario}{
		Permite hacer el registro de un comentario para un servicio turístico, así mismo se podrá registrar una calificación. Para hacer el registro de un comentario se debe de cumplir la regla \cdtIdRef{BR-003}{Tiempo transcurrido para registrar comentario}, ademas de encontrarse dentro de la zona turística a la que pertenece el servicio turístico.
	
	}
		\UCitem{Versión}{\color{Gray}1.0}
		\UCitem{Actor}{\cdtRef{actor:turista}{Turista}}
		\UCitem{Propósito}{Registrar un comentario y una calificación a un servicio turístico.}
		\UCitem{Entradas}{
			\begin{UClist}
				\UCli Zona turística
				\UCli Posicionamiento
				\UCli Tipo de servicio
				\UCli Nombre de servicio
				\UCli Puntuación del servicio
				\UCli Comentarios del servicio
			\end{UClist}
		}
		\UCitem{Origen}{
			\begin{UClist}
				\UCli \ioSistema
				\UCli \ioEscribir
			\end{UClist}
		}
		\UCitem{Salidas}{\cdtIdRef{MSG1}{Operación exitosa}}
		\UCitem{Precondiciones}{Haber cumplido con el tiempo transcurrido para registro de comentario.}
		\UCitem{Postcondiciones}{Se registrará el comentario y la calificación para el servicio turístico.}
		\UCitem{Reglas de negocio}{\cdtIdRef{BR-003}{Tiempo transcurrido para registrar comentario}}
		\UCitem{Errores}{
			\begin{UClist}
				\UCli \UCerr{Uno}{Cuando no ha transcurrido el tiempo requerido para registrar comentario,}{muestra el mensaje \cdtIdRef{MSG}{No se puede registrar comentario aún} y regresa al paso \ref{SE-CU2.1.2.1:Pantalla} de la trayectoria principal.}
				\UCli \UCerr{Dos}{Cuando no se puede llevar a cabo la operación de manera exitosa,}{muestra el mensaje \cdtIdRef{MSG2}{Operación fallida} y termina el caso de uso.}
			\end{UClist}	
		}
		\UCitem{Tipo}{Extiende de \cdtIdRef{LR-CU2.1.2.1}{Consultar Información y Estadísticas Turísticas}.}
	\end{UseCase}
%--------------------------------------
	\begin{UCtrayectoria}
		 
		\UCpaso [\UCactor] Solicita registrar un comentario al servicio turístico tocando el botón \IUbutton{Comentar} en la pantalla \IUref{SE-IU2.1.2.1a}{Consultar información y estadísticas turísticas}.
		
		\UCpaso Verifica que haya transcurrido el tiempo requerido para registrar comentario, con base en la regla \cdtIdRef{BR-003}{Tiempo transcurrido para registrar comentario}. \refErr{Uno}
		
		\UCpaso Muestra la pantalla \IUref{SE-IU2.1.2.1.1}{Registrar comentario} con el formulario para registrar el comentario.
		
		\UCpaso [\UCactor] Ingresa los datos solicitados.
		
		\UCpaso [\UCactor] Solicita finalizar la operación tocando el botón \IUbutton{Enviar}.
		
		\UCpaso Asocia el comentario y la calificación registrada. \refErr{Dos}
		
		\UCpaso Muestra el mensaje \cdtIdRef{MSG1}{Operación exitosa}
	\end{UCtrayectoria}

% \IUref{IUAdmPS}{Administrar Planta de Selección}
% \IUref{IUModPS}{Modificar Planta de Selección}
% \IUref{IUEliPS}{Eliminar Planta de Selección}

% 


% Copie este bloque por cada caso de uso:
%-------------------------------------- COMIENZA descripción del caso de uso.

%\begin{UseCase}[archivo de imágen]{UCX}{Nombre del Caso de uso}{
%--------------------------------------
	\begin{UseCase}{SE-CU2.1.2.1.1}{Registrar Comentario}{
		Permite hacer el registro de un comentario para un servicio turístico, así mismo se podrá registrar una calificación. Para hacer el registro de un comentario se debe de cumplir la regla \cdtIdRef{BR-003}{Tiempo transcurrido para registrar comentario}, ademas de encontrarse dentro de la zona turística a la que pertenece el servicio turístico.
	
	}
		\UCitem{Versión}{\color{Gray}1.0}
		\UCitem{Actor}{\cdtRef{actor:turista}{Turista}}
		\UCitem{Propósito}{Registrar un comentario y una calificación a un servicio turístico.}
		\UCitem{Entradas}{
			\begin{UClist}
				\UCli Zona turística
				\UCli Posicionamiento
				\UCli Tipo de servicio
				\UCli Nombre de servicio
				\UCli Puntuación del servicio
				\UCli Comentarios del servicio
			\end{UClist}
		}
		\UCitem{Origen}{
			\begin{UClist}
				\UCli \ioSistema
				\UCli \ioEscribir
			\end{UClist}
		}
		\UCitem{Salidas}{\cdtIdRef{MSG1}{Operación exitosa}}
		\UCitem{Precondiciones}{Haber cumplido con el tiempo transcurrido para registro de comentario.}
		\UCitem{Postcondiciones}{Se registrará el comentario y la calificación para el servicio turístico.}
		\UCitem{Reglas de negocio}{\cdtIdRef{BR-003}{Tiempo transcurrido para registrar comentario}}
		\UCitem{Errores}{
			\begin{UClist}
				\UCli \UCerr{Uno}{Cuando no ha transcurrido el tiempo requerido para registrar comentario,}{muestra el mensaje \cdtIdRef{MSG}{No se puede registrar comentario aún} y regresa al paso \ref{SE-CU2.1.2.1:Pantalla} de la trayectoria principal.}
				\UCli \UCerr{Dos}{Cuando no se puede llevar a cabo la operación de manera exitosa,}{muestra el mensaje \cdtIdRef{MSG2}{Operación fallida} y termina el caso de uso.}
			\end{UClist}	
		}
		\UCitem{Tipo}{Extiende de \cdtIdRef{LR-CU2.1.2.1}{Consultar Información y Estadísticas Turísticas}.}
	\end{UseCase}
%--------------------------------------
	\begin{UCtrayectoria}
		 
		\UCpaso [\UCactor] Solicita registrar un comentario al servicio turístico tocando el botón \IUbutton{Comentar} en la pantalla \IUref{SE-IU2.1.2.1a}{Consultar información y estadísticas turísticas}.
		
		\UCpaso Verifica que haya transcurrido el tiempo requerido para registrar comentario, con base en la regla \cdtIdRef{BR-003}{Tiempo transcurrido para registrar comentario}. \refErr{Uno}
		
		\UCpaso Muestra la pantalla \IUref{SE-IU2.1.2.1.1}{Registrar comentario} con el formulario para registrar el comentario.
		
		\UCpaso [\UCactor] Ingresa los datos solicitados.
		
		\UCpaso [\UCactor] Solicita finalizar la operación tocando el botón \IUbutton{Enviar}.
		
		\UCpaso Asocia el comentario y la calificación registrada. \refErr{Dos}
		
		\UCpaso Muestra el mensaje \cdtIdRef{MSG1}{Operación exitosa}
	\end{UCtrayectoria}

% \IUref{IUAdmPS}{Administrar Planta de Selección}
% \IUref{IUModPS}{Modificar Planta de Selección}
% \IUref{IUEliPS}{Eliminar Planta de Selección}

% 


% Copie este bloque por cada caso de uso:
%-------------------------------------- COMIENZA descripción del caso de uso.

%\begin{UseCase}[archivo de imágen]{UCX}{Nombre del Caso de uso}{
%--------------------------------------
	\begin{UseCase}{SE-CU2.1.2.1.1}{Registrar Comentario}{
		Permite hacer el registro de un comentario para un servicio turístico, así mismo se podrá registrar una calificación. Para hacer el registro de un comentario se debe de cumplir la regla \cdtIdRef{BR-003}{Tiempo transcurrido para registrar comentario}, ademas de encontrarse dentro de la zona turística a la que pertenece el servicio turístico.
	
	}
		\UCitem{Versión}{\color{Gray}1.0}
		\UCitem{Actor}{\cdtRef{actor:turista}{Turista}}
		\UCitem{Propósito}{Registrar un comentario y una calificación a un servicio turístico.}
		\UCitem{Entradas}{
			\begin{UClist}
				\UCli Zona turística
				\UCli Posicionamiento
				\UCli Tipo de servicio
				\UCli Nombre de servicio
				\UCli Puntuación del servicio
				\UCli Comentarios del servicio
			\end{UClist}
		}
		\UCitem{Origen}{
			\begin{UClist}
				\UCli \ioSistema
				\UCli \ioEscribir
			\end{UClist}
		}
		\UCitem{Salidas}{\cdtIdRef{MSG1}{Operación exitosa}}
		\UCitem{Precondiciones}{Haber cumplido con el tiempo transcurrido para registro de comentario.}
		\UCitem{Postcondiciones}{Se registrará el comentario y la calificación para el servicio turístico.}
		\UCitem{Reglas de negocio}{\cdtIdRef{BR-003}{Tiempo transcurrido para registrar comentario}}
		\UCitem{Errores}{
			\begin{UClist}
				\UCli \UCerr{Uno}{Cuando no ha transcurrido el tiempo requerido para registrar comentario,}{muestra el mensaje \cdtIdRef{MSG}{No se puede registrar comentario aún} y regresa al paso \ref{SE-CU2.1.2.1:Pantalla} de la trayectoria principal.}
				\UCli \UCerr{Dos}{Cuando no se puede llevar a cabo la operación de manera exitosa,}{muestra el mensaje \cdtIdRef{MSG2}{Operación fallida} y termina el caso de uso.}
			\end{UClist}	
		}
		\UCitem{Tipo}{Extiende de \cdtIdRef{LR-CU2.1.2.1}{Consultar Información y Estadísticas Turísticas}.}
	\end{UseCase}
%--------------------------------------
	\begin{UCtrayectoria}
		 
		\UCpaso [\UCactor] Solicita registrar un comentario al servicio turístico tocando el botón \IUbutton{Comentar} en la pantalla \IUref{SE-IU2.1.2.1a}{Consultar información y estadísticas turísticas}.
		
		\UCpaso Verifica que haya transcurrido el tiempo requerido para registrar comentario, con base en la regla \cdtIdRef{BR-003}{Tiempo transcurrido para registrar comentario}. \refErr{Uno}
		
		\UCpaso Muestra la pantalla \IUref{SE-IU2.1.2.1.1}{Registrar comentario} con el formulario para registrar el comentario.
		
		\UCpaso [\UCactor] Ingresa los datos solicitados.
		
		\UCpaso [\UCactor] Solicita finalizar la operación tocando el botón \IUbutton{Enviar}.
		
		\UCpaso Asocia el comentario y la calificación registrada. \refErr{Dos}
		
		\UCpaso Muestra el mensaje \cdtIdRef{MSG1}{Operación exitosa}
	\end{UCtrayectoria}

% \IUref{IUAdmPS}{Administrar Planta de Selección}
% \IUref{IUModPS}{Modificar Planta de Selección}
% \IUref{IUEliPS}{Eliminar Planta de Selección}

% 


% Copie este bloque por cada caso de uso:
%-------------------------------------- COMIENZA descripción del caso de uso.

%\begin{UseCase}[archivo de imágen]{UCX}{Nombre del Caso de uso}{
%--------------------------------------
	\begin{UseCase}{SE-CU2.1.2.1.1}{Registrar Comentario}{
		Permite hacer el registro de un comentario para un servicio turístico, así mismo se podrá registrar una calificación. Para hacer el registro de un comentario se debe de cumplir la regla \cdtIdRef{BR-003}{Tiempo transcurrido para registrar comentario}, ademas de encontrarse dentro de la zona turística a la que pertenece el servicio turístico.
	
	}
		\UCitem{Versión}{\color{Gray}1.0}
		\UCitem{Actor}{\cdtRef{actor:turista}{Turista}}
		\UCitem{Propósito}{Registrar un comentario y una calificación a un servicio turístico.}
		\UCitem{Entradas}{
			\begin{UClist}
				\UCli Zona turística
				\UCli Posicionamiento
				\UCli Tipo de servicio
				\UCli Nombre de servicio
				\UCli Puntuación del servicio
				\UCli Comentarios del servicio
			\end{UClist}
		}
		\UCitem{Origen}{
			\begin{UClist}
				\UCli \ioSistema
				\UCli \ioEscribir
			\end{UClist}
		}
		\UCitem{Salidas}{\cdtIdRef{MSG1}{Operación exitosa}}
		\UCitem{Precondiciones}{Haber cumplido con el tiempo transcurrido para registro de comentario.}
		\UCitem{Postcondiciones}{Se registrará el comentario y la calificación para el servicio turístico.}
		\UCitem{Reglas de negocio}{\cdtIdRef{BR-003}{Tiempo transcurrido para registrar comentario}}
		\UCitem{Errores}{
			\begin{UClist}
				\UCli \UCerr{Uno}{Cuando no ha transcurrido el tiempo requerido para registrar comentario,}{muestra el mensaje \cdtIdRef{MSG}{No se puede registrar comentario aún} y regresa al paso \ref{SE-CU2.1.2.1:Pantalla} de la trayectoria principal.}
				\UCli \UCerr{Dos}{Cuando no se puede llevar a cabo la operación de manera exitosa,}{muestra el mensaje \cdtIdRef{MSG2}{Operación fallida} y termina el caso de uso.}
			\end{UClist}	
		}
		\UCitem{Tipo}{Extiende de \cdtIdRef{LR-CU2.1.2.1}{Consultar Información y Estadísticas Turísticas}.}
	\end{UseCase}
%--------------------------------------
	\begin{UCtrayectoria}
		 
		\UCpaso [\UCactor] Solicita registrar un comentario al servicio turístico tocando el botón \IUbutton{Comentar} en la pantalla \IUref{SE-IU2.1.2.1a}{Consultar información y estadísticas turísticas}.
		
		\UCpaso Verifica que haya transcurrido el tiempo requerido para registrar comentario, con base en la regla \cdtIdRef{BR-003}{Tiempo transcurrido para registrar comentario}. \refErr{Uno}
		
		\UCpaso Muestra la pantalla \IUref{SE-IU2.1.2.1.1}{Registrar comentario} con el formulario para registrar el comentario.
		
		\UCpaso [\UCactor] Ingresa los datos solicitados.
		
		\UCpaso [\UCactor] Solicita finalizar la operación tocando el botón \IUbutton{Enviar}.
		
		\UCpaso Asocia el comentario y la calificación registrada. \refErr{Dos}
		
		\UCpaso Muestra el mensaje \cdtIdRef{MSG1}{Operación exitosa}
	\end{UCtrayectoria}



