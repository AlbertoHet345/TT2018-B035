\begin{titlepage} %PAGINA PARA PORTADA

\setlength{\unitlength}{1 cm}
\thispagestyle{empty}


%\begin{center}
%	\textbf{{{\LARGE Instituto Politécnico Nacional}} \\[0.5cm]
%	{\Large Escuela Superior de Cómputo}}           \\[1.5cm]
%\end{center}

\begin{picture}(17,0)
	\put(-0.2,-2){\includegraphics[width=2.2cm,height=2.5cm]{images/logos/IPN}} %mas largo
	\put(13.4,-2){\includegraphics[width=3cm,height=2.5cm]{images/logos/escom.png}} %menos ancho
\end{picture}

\begin{center}
	\textbf{{{\LARGE Instituto Politécnico Nacional}} \\[0.5cm]
	{\Large Escuela Superior de Cómputo}}           \\[1.5cm]
	%{\large\textit{Trabajo Terminal}}                 \\[0.5cm]
	%{\Large \textbf{ ``Clasificación de noticias de diarios de circulación nacional mediante aprendizaje automático"}}  \\[0.5cm]
	%{\LARGE 2017-A042}\\[2cm]
	%{\Large Que para cumplir con la opción de titulación curricular en la carrera de}  \\[1cm]
	%{\LARGE \textbf{``Ingeniería en Sistemas Computacionales"}} \\[2cm]
	%{\Large \textit{Presentan}}                 \\[0.5cm]
	%{\large García Molina José Alejandro\\[0.3cm]}
	%{\large Sanchéz Ramírez Miguel Angel\\[0.3cm]}
	%{\large Ramírez Roque Luis Enrique\\[1cm]}
	%{\Large \textit{Directores}}                 \\[0.5cm]
	%{\large Juárez Gambino Joel Omar \tab[2cm]  García Mendoza Consuelo Varinia\\[1cm]}
	%\today
\end{center}

\noindent
	\Large No. de TT: 2018-B035  \hfill \large \today \\%\\[0.3cm]	

\begin{center}
	\Large Documento Técnico\\[0.7cm]
\end{center}

\begin{center}
	\textbf{\Large ''Asistente turístico basado en trazado de áreas geográficas de interés''}\\[0.7cm]
\end{center}

\begin{center}
	\Large Presentan:\\[0.3cm]
	\textbf{Alberto García Paul}\footnote{albertopaul3@icloud.com}\\[0.2cm]
	\textbf{Isaac Abraham Meza Sánchez}\footnote{isaac\_ims@hotmail.com}\\[0.8cm]
\end{center}

\begin{center}
	\textit{\Large Directores} \\[0.3cm]
	\textbf{M. en C. Chadwick Carreto Arellano} \qquad\qquad  \textbf{Dra. Elena Fabiola Ruíz Ledezma}\\[0.8cm]
\end{center}

\begin{center}
	{\Large \textbf{RESUMEN}}\\[0.2cm]
\end{center}
{\setlength{\parindent}{0pt}
En este trabajo terminal se propone clasificar mediante técnicas de aprendizaje automático,
noticias de diarios de circulación nacional en las diferentes secciones en que en estos se
dividen, por ejemplo: cultura, deportes, política. La tarea de clasificar un diario en secciones se
realiza manualmente, lo cual implica tiempo y esfuerzo por parte del editor.
El trabajo contempla recolectar noticias de tres diarios de circulación nacional en las cuales el
editor ya ha marcado a qué sección pertenecen. De estas noticias se extraerán diferentes
características que servirán para utilizarlas en algoritmos de aprendizaje automático. Se
probarán diferentes técnicas de extracción de características junto con diferentes algoritmos de
aprendizaje automático, y al final se seleccionarán aquellos que obtengan mejores resultados.
Con las técnicas seleccionadas, se podrán clasificar nuevas noticias en las secciones
correspondientes de los diarios de forma automática.\\[0.5CM]
Palabras clave: Turismo Interactivo, e-Turismo, Internet de las cosas, Geocercas, Servicios Turísticos

}
\begin{figure}[htbp]
	\begin{center}
		\includegraphics[scale=.4]{images/carta}
	\end{center}
\end{figure}
\newpage
\thispagestyle{empty}
\begin{center}
	{\Large \textbf{Advertencia}}\\[3cm]
\end{center}

\begin{center}
{\fboxsep 12pt \fboxrule 1pt
	\fbox{%
	\begin{minipage}[c][10cm][c]{0.8\linewidth}
\textit{“Este documento contiene información desarrollada por la Escuela Superior de Cómputo del Instituto Politécnico Nacional, a partir de datos y documentos con derecho de propiedad y por lo tanto, su uso quedará restringido a las aplicaciones que explícitamente se convengan.”}\\

La aplicación no convenida exime a la escuela su responsabilidad técnica y da lugar a las consecuencias legales que para tal efecto se determinen.\\

Información adicional sobre este reporte técnico podrá obtenerse en:\\

La Subdirección Académica de la Escuela Superior de Cómputo del Instituto Politécnico Nacional, situada en Av. Juan de Dios Bátiz s/n Teléfono: 57296000, extensión 52000.\\
	\end{minipage}
	}\hfill
}
\end{center}

\end{titlepage}