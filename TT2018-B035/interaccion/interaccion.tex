\chapter{Modelo de Ineracción}

En el presente capítulo se describirán todos aquellos componentes en los que el usuario final tendrá interacción con el sistema, para exista esta interacción se llevó a cabo el proceso de diseño en el cual, con base en los requerimientos descritos en capítulos anteriores, se plateó un modelo que cumpla con dichos requerimientos. \\

En este proceso se llevó a cabo el diseño de pantallas de la aplicación, que es el medio por el cual el usuario tendrá interacción con el sistema. Así mismo se diseñaron mensajes que serán mostrados al usuario conforme éste vaya interactuando con el sistema. \\

De igual forma cada uno de estos diseños cuenta con una nomenclatura para que sea más sencilla su identificación. A continuación se describe la nomenclatura utilizada: 

\begin{center}
	\Huge{\textit{Iniciales\_módulo}-IU\textit{n}: \textit{nombre\_pantalla}}
\end{center}

Donde: 

\begin{itemize}
	\item \textbf{iniciales\_módulo}: significa que son las iniciales del módulo en el que se encuentra el caso de uso, los módulos son: 
	\begin{itemize}
		\item RE: Registro
		\item LR: Localización de rutas
		\item PO: Posicionamiento
		\item II: Interacción de la información
		\item SE: Servicios
		\item IR: Información y representación de estadísticas turísticas
		\item RA: Registro de área turística
	\end{itemize}
	
	\item \textbf{n}: es el número de pantalla.
	
	\item \textbf{nombre\_pantalla}: es el nombre de la pantalla.
\end{itemize}

Cabe mencionar que el nombre de las pantallas estarán empatadas con el nombre del caso de uso en que se utiliza, sin embargo si un caso de uso utiliza más de una pantalla simplemente se agregará una literal al lado del número de esta. Por ejemplo: 

\begin{center}
	RE-IU1a: Registro por correo\\
	RE-IU1b: Registro por Facebook\\
	RE-IU1c: Registro por Google\\
\end{center}

Al igual que los casos de uso, las pantallas cuentan con una descripción detallada. Dicha descripción se encuentra compuesta de la siguiente manera:

\begin{enumerate}
	\item Objetivo
	\item Diseño
\end{enumerate}

En el objetivo se detallará cual es el objetivo que tiene la pantalla, muchas veces éste es similar a la descripción del caso de uso, de igual manera en este se detallaran las entradas y salidas, así como los mensajes y comandos a utilizar. \\

En el diseño se describirá el por qué se diseñó de tal manera la pantalla, así mismo se detallará el comportamiento que ésta podrá tener durante la interacción con el usuario, detallando comportamientos que no pueden ser descritos en el caso de uso.\\

A continuación se describirán las pantallas para el sistema:
\newpage
\section{LR-IU2.1 Visualizar Información de Zona Turística}

\subsection{Objetivo}
Permite visualizar la información de alguna zona turística que sea del interés del usuario.

\subsection{Diseño}
La figura \ref{LR-IU2.1a} muestra la pantalla \IUref{LR-IU2.1a}{Zona Turística} la cual contiene la descripción de la zona turística seleccionada. Así mismo en la pantalla se encuentran los botones \IUbutton{Ver mapa} y \IUbutton{Ver Servicios}. Al tocar el botón \IUbutton{Ver mapa} se mostrará la pantalla \IUref{LR-IU2.1b}{Mapa de Zona Turística}.

\IUfig[.2]{../CasosDeUso/cu2_1/images/LR-IU2_1}{LR-IU2.1a}{Zona Turística}


La Figura \ref{LR-IU2.1b} contine la pantalla \IUref{LR-IU2.1b}{Mapa de Zona Turística} en la que se muestra un mapa de la zona turística con todos sus servicios.

\IUfig[.2]{../CasosDeUso/cu2_1/images/LR-IU2_1b}{LR-IU2.1b}{Mapa de Zona Turística}
\newpage
\section{LR-IU2.1 Visualizar Información de Zona Turística}

\subsection{Objetivo}
Permite visualizar la información de alguna zona turística que sea del interés del usuario.

\subsection{Diseño}
La figura \ref{LR-IU2.1a} muestra la pantalla \IUref{LR-IU2.1a}{Zona Turística} la cual contiene la descripción de la zona turística seleccionada. Así mismo en la pantalla se encuentran los botones \IUbutton{Ver mapa} y \IUbutton{Ver Servicios}. Al tocar el botón \IUbutton{Ver mapa} se mostrará la pantalla \IUref{LR-IU2.1b}{Mapa de Zona Turística}.

\IUfig[.2]{../CasosDeUso/cu2_1/images/LR-IU2_1}{LR-IU2.1a}{Zona Turística}


La Figura \ref{LR-IU2.1b} contine la pantalla \IUref{LR-IU2.1b}{Mapa de Zona Turística} en la que se muestra un mapa de la zona turística con todos sus servicios.

\IUfig[.2]{../CasosDeUso/cu2_1/images/LR-IU2_1b}{LR-IU2.1b}{Mapa de Zona Turística}
\newpage
\section{LR-IU2.1 Visualizar Información de Zona Turística}

\subsection{Objetivo}
Permite visualizar la información de alguna zona turística que sea del interés del usuario.

\subsection{Diseño}
La figura \ref{LR-IU2.1a} muestra la pantalla \IUref{LR-IU2.1a}{Zona Turística} la cual contiene la descripción de la zona turística seleccionada. Así mismo en la pantalla se encuentran los botones \IUbutton{Ver mapa} y \IUbutton{Ver Servicios}. Al tocar el botón \IUbutton{Ver mapa} se mostrará la pantalla \IUref{LR-IU2.1b}{Mapa de Zona Turística}.

\IUfig[.2]{../CasosDeUso/cu2_1/images/LR-IU2_1}{LR-IU2.1a}{Zona Turística}


La Figura \ref{LR-IU2.1b} contine la pantalla \IUref{LR-IU2.1b}{Mapa de Zona Turística} en la que se muestra un mapa de la zona turística con todos sus servicios.

\IUfig[.2]{../CasosDeUso/cu2_1/images/LR-IU2_1b}{LR-IU2.1b}{Mapa de Zona Turística}
\newpage
\section{LR-IU2.1 Visualizar Información de Zona Turística}

\subsection{Objetivo}
Permite visualizar la información de alguna zona turística que sea del interés del usuario.

\subsection{Diseño}
La figura \ref{LR-IU2.1a} muestra la pantalla \IUref{LR-IU2.1a}{Zona Turística} la cual contiene la descripción de la zona turística seleccionada. Así mismo en la pantalla se encuentran los botones \IUbutton{Ver mapa} y \IUbutton{Ver Servicios}. Al tocar el botón \IUbutton{Ver mapa} se mostrará la pantalla \IUref{LR-IU2.1b}{Mapa de Zona Turística}.

\IUfig[.2]{../CasosDeUso/cu2_1/images/LR-IU2_1}{LR-IU2.1a}{Zona Turística}


La Figura \ref{LR-IU2.1b} contine la pantalla \IUref{LR-IU2.1b}{Mapa de Zona Turística} en la que se muestra un mapa de la zona turística con todos sus servicios.

\IUfig[.2]{../CasosDeUso/cu2_1/images/LR-IU2_1b}{LR-IU2.1b}{Mapa de Zona Turística}
\newpage
\section{LR-IU2.1 Visualizar Información de Zona Turística}

\subsection{Objetivo}
Permite visualizar la información de alguna zona turística que sea del interés del usuario.

\subsection{Diseño}
La figura \ref{LR-IU2.1a} muestra la pantalla \IUref{LR-IU2.1a}{Zona Turística} la cual contiene la descripción de la zona turística seleccionada. Así mismo en la pantalla se encuentran los botones \IUbutton{Ver mapa} y \IUbutton{Ver Servicios}. Al tocar el botón \IUbutton{Ver mapa} se mostrará la pantalla \IUref{LR-IU2.1b}{Mapa de Zona Turística}.

\IUfig[.2]{../CasosDeUso/cu2_1/images/LR-IU2_1}{LR-IU2.1a}{Zona Turística}


La Figura \ref{LR-IU2.1b} contine la pantalla \IUref{LR-IU2.1b}{Mapa de Zona Turística} en la que se muestra un mapa de la zona turística con todos sus servicios.

\IUfig[.2]{../CasosDeUso/cu2_1/images/LR-IU2_1b}{LR-IU2.1b}{Mapa de Zona Turística}
\newpage
\section{LR-IU2.1 Visualizar Información de Zona Turística}

\subsection{Objetivo}
Permite visualizar la información de alguna zona turística que sea del interés del usuario.

\subsection{Diseño}
La figura \ref{LR-IU2.1a} muestra la pantalla \IUref{LR-IU2.1a}{Zona Turística} la cual contiene la descripción de la zona turística seleccionada. Así mismo en la pantalla se encuentran los botones \IUbutton{Ver mapa} y \IUbutton{Ver Servicios}. Al tocar el botón \IUbutton{Ver mapa} se mostrará la pantalla \IUref{LR-IU2.1b}{Mapa de Zona Turística}.

\IUfig[.2]{../CasosDeUso/cu2_1/images/LR-IU2_1}{LR-IU2.1a}{Zona Turística}


La Figura \ref{LR-IU2.1b} contine la pantalla \IUref{LR-IU2.1b}{Mapa de Zona Turística} en la que se muestra un mapa de la zona turística con todos sus servicios.

\IUfig[.2]{../CasosDeUso/cu2_1/images/LR-IU2_1b}{LR-IU2.1b}{Mapa de Zona Turística}
\newpage
\section{LR-IU2.1 Visualizar Información de Zona Turística}

\subsection{Objetivo}
Permite visualizar la información de alguna zona turística que sea del interés del usuario.

\subsection{Diseño}
La figura \ref{LR-IU2.1a} muestra la pantalla \IUref{LR-IU2.1a}{Zona Turística} la cual contiene la descripción de la zona turística seleccionada. Así mismo en la pantalla se encuentran los botones \IUbutton{Ver mapa} y \IUbutton{Ver Servicios}. Al tocar el botón \IUbutton{Ver mapa} se mostrará la pantalla \IUref{LR-IU2.1b}{Mapa de Zona Turística}.

\IUfig[.2]{../CasosDeUso/cu2_1/images/LR-IU2_1}{LR-IU2.1a}{Zona Turística}


La Figura \ref{LR-IU2.1b} contine la pantalla \IUref{LR-IU2.1b}{Mapa de Zona Turística} en la que se muestra un mapa de la zona turística con todos sus servicios.

\IUfig[.2]{../CasosDeUso/cu2_1/images/LR-IU2_1b}{LR-IU2.1b}{Mapa de Zona Turística}
\newpage
\section{LR-IU2.1 Visualizar Información de Zona Turística}

\subsection{Objetivo}
Permite visualizar la información de alguna zona turística que sea del interés del usuario.

\subsection{Diseño}
La figura \ref{LR-IU2.1a} muestra la pantalla \IUref{LR-IU2.1a}{Zona Turística} la cual contiene la descripción de la zona turística seleccionada. Así mismo en la pantalla se encuentran los botones \IUbutton{Ver mapa} y \IUbutton{Ver Servicios}. Al tocar el botón \IUbutton{Ver mapa} se mostrará la pantalla \IUref{LR-IU2.1b}{Mapa de Zona Turística}.

\IUfig[.2]{../CasosDeUso/cu2_1/images/LR-IU2_1}{LR-IU2.1a}{Zona Turística}


La Figura \ref{LR-IU2.1b} contine la pantalla \IUref{LR-IU2.1b}{Mapa de Zona Turística} en la que se muestra un mapa de la zona turística con todos sus servicios.

\IUfig[.2]{../CasosDeUso/cu2_1/images/LR-IU2_1b}{LR-IU2.1b}{Mapa de Zona Turística}
%% !TeX spellcheck = <none>
% \IUref{IUAdmPS}{Administrar Planta de Selección}
% \IUref{IUModPS}{Modificar Planta de Selección}
% \IUref{IUEliPS}{Eliminar Planta de Selección}

% 


% Copie este bloque por cada caso de uso:
%-------------------------------------- COMIENZA descripción del caso de uso.

%\begin{UseCase}[archivo de imágen]{UCX}{Nombre del Caso de uso}{
%--------------------------------------
	\begin{UseCase}{CU6}{Registrar Comentario}{
		Este caso de uso permite al usuario registrar un comentario y una puntuación a un servicio turístico seleccionado que se encuentre dentro de un área registrada. Cabe destacar que para poder acceder a este caso de uso es necesario que se valide la regla de negocio XXXXX .
	}
		\UCitem{Versión}{\color{Gray}1.0}
		\UCitem{Actor}{\hyperlink{Usuario}{Usuario}}
		\UCitem{Propósito}{Registrar un comentario y puntuación.}
		\UCitem{Entradas}{Comentario y Puntuación}
		\UCitem{Origen}{Teclado}
		\UCitem{Salidas}{N.A.}
		\UCitem{Precondiciones}{Cumplir con la regla de negocio XXXXXX}
		\UCitem{Postcondiciones}{Quedará el comentario y la puntuación, asociada al servicio turístico}
		\UCitem{Errores}{}
		\UCitem{Tipo}{Caso de uso Cuaternario}
		\UCitem{Observaciones}{}
	\end{UseCase}
%--------------------------------------
	\begin{UCtrayectoria} 
		
		\UCpaso[\UCactor] Da clic en el botón \IUbutton{Comentar} de la pantalla  \IUref{IU8}{Principal}.
		
		\UCpaso Obtiene la descripción, puntuación y los comentarios del servicio seleccionado.
		
		\UCpaso Verifica que el usuario tenga permitido comentar mediante la regla de negocio XXXXXXXXXX. \Trayref{A}.
		
		\UCpaso Despliega la pantalla \IUref{IU8}{Principal} con los datos asociados al servicio, así como habilitados los botones \IUbutton{Ver Mapa} y \IUbutton{Comentar}.
		
		\UCpaso[] Termina el caso de uso.
		
	\end{UCtrayectoria}

%--------------------------------------		
		\begin{UCtrayectoriaA}{A}{El usuario no tiene permitido realizar un comentario.}
			
		\UCpaso Despliega la pantalla \IUref{IU8}{Principal} con los datos asociados al servicio y únicamente habilitado el botón \IUbutton{Ver Mapa}.
		
		\UCpaso[] Termina el caso de uso.
		
	\end{UCtrayectoriaA}
	
	
%% !TeX spellcheck = <none>
% \IUref{IUAdmPS}{Administrar Planta de Selección}
% \IUref{IUModPS}{Modificar Planta de Selección}
% \IUref{IUEliPS}{Eliminar Planta de Selección}

% 


% Copie este bloque por cada caso de uso:
%-------------------------------------- COMIENZA descripción del caso de uso.

%\begin{UseCase}[archivo de imágen]{UCX}{Nombre del Caso de uso}{
%--------------------------------------
	\begin{UseCase}{CU6}{Registrar Comentario}{
		Este caso de uso permite al usuario registrar un comentario y una puntuación a un servicio turístico seleccionado que se encuentre dentro de un área registrada. Cabe destacar que para poder acceder a este caso de uso es necesario que se valide la regla de negocio XXXXX .
	}
		\UCitem{Versión}{\color{Gray}1.0}
		\UCitem{Actor}{\hyperlink{Usuario}{Usuario}}
		\UCitem{Propósito}{Registrar un comentario y puntuación.}
		\UCitem{Entradas}{Comentario y Puntuación}
		\UCitem{Origen}{Teclado}
		\UCitem{Salidas}{N.A.}
		\UCitem{Precondiciones}{Cumplir con la regla de negocio XXXXXX}
		\UCitem{Postcondiciones}{Quedará el comentario y la puntuación, asociada al servicio turístico}
		\UCitem{Errores}{}
		\UCitem{Tipo}{Caso de uso Cuaternario}
		\UCitem{Observaciones}{}
	\end{UseCase}
%--------------------------------------
	\begin{UCtrayectoria} 
		
		\UCpaso[\UCactor] Da clic en el botón \IUbutton{Comentar} de la pantalla  \IUref{IU8}{Principal}.
		
		\UCpaso Obtiene la descripción, puntuación y los comentarios del servicio seleccionado.
		
		\UCpaso Verifica que el usuario tenga permitido comentar mediante la regla de negocio XXXXXXXXXX. \Trayref{A}.
		
		\UCpaso Despliega la pantalla \IUref{IU8}{Principal} con los datos asociados al servicio, así como habilitados los botones \IUbutton{Ver Mapa} y \IUbutton{Comentar}.
		
		\UCpaso[] Termina el caso de uso.
		
	\end{UCtrayectoria}

%--------------------------------------		
		\begin{UCtrayectoriaA}{A}{El usuario no tiene permitido realizar un comentario.}
			
		\UCpaso Despliega la pantalla \IUref{IU8}{Principal} con los datos asociados al servicio y únicamente habilitado el botón \IUbutton{Ver Mapa}.
		
		\UCpaso[] Termina el caso de uso.
		
	\end{UCtrayectoriaA}
	
	
