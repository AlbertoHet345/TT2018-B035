% \IUref{IUAdmPS}{Administrar Planta de Selección}
% \IUref{IUModPS}{Modificar Planta de Selección}
% \IUref{IUEliPS}{Eliminar Planta de Selección}

% 


% Copie este bloque por cada caso de uso:
%-------------------------------------- COMIENZA descripción del caso de uso.

%\begin{UseCase}[archivo de imágen]{UCX}{Nombre del Caso de uso}{
%--------------------------------------
	\begin{UseCase}{CU4}{servicios Turísticos}{
		Este caso de uso permite al usuario visualizar el listado de los servicios turísticos ofrecidos en un área turística específica, catalogándolos por tipo de servicio o bien por un filtro donde el usuario puede buscar por nombre el servicio requerido. 
	}
		\UCitem{Versión}{\color{Gray}1.0}
		\UCitem{Actor}{\hyperlink{Usuario}{Usuario}}
		\UCitem{Propósito}{Visualizar los servicios turísticos disponibles en el área.}
		\UCitem{Entradas}{N.A.}
		\UCitem{Origen}{N.A.}
		\UCitem{Salidas}{Listado de Servicios Turísticos de un área}
		\UCitem{Precondiciones}{N.A.}
		\UCitem{Postcondiciones}{N.A.}
		\UCitem{Errores}{}
		\UCitem{Tipo}{Caso de uso Secundario}
		\UCitem{Observaciones}{}
	\end{UseCase}
%--------------------------------------
	\begin{UCtrayectoria} 
		\UCpaso[\UCactor] Da clic en el botón \IUbutton{Servicios turísticos en el área} de la pantalla \IUref{IU8}{Principal}.
		
		\UCpaso Obtiene el catálogo de los servicios turísticos registrados en el área.
		
		\UCpaso Despliega la pantalla \IUref{IU8}{Principal} con los botones \IUbutton{Generar ruta a partir de la ubicación actual}, \IUbutton{Servicios turísticos en el área} habilitados.

		\UCpaso[] Termina el caso de uso.
		
	\end{UCtrayectoria}

%--------------------------------------		
		\begin{UCtrayectoriaA}{A}{El usuario no se encuentra ubicado en alguna zona turística registrada}
		\UCpaso Deshabilita el botón \IUbutton{Generar ruta a partir de la ubicación actual}.
		
		\UCpaso Despliega la pantalla \IUref{IU8}{Principal} unicamente con el botón \IUbutton{Servicios turísticos cerca de mi} habilitado.
		
		\UCpaso[] Termina el caso de uso.
		
	\end{UCtrayectoriaA}
	
	