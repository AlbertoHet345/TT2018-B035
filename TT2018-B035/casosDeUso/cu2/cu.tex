% \IUref{IUAdmPS}{Administrar Planta de Selección}
% \IUref{IUModPS}{Modificar Planta de Selección}
% \IUref{IUEliPS}{Eliminar Planta de Selección}

% 


% Copie este bloque por cada caso de uso:
%-------------------------------------- COMIENZA descripción del caso de uso.

%\begin{UseCase}[archivo de imágen]{UCX}{Nombre del Caso de uso}{
%--------------------------------------
	\begin{UseCase}{LR-CU2}{Obtener Zonas Turísticas}{
		
		Permite obtener y visualizar una lista con las zonas turísticas que han sido visitadas por el turista, así mismo se puede obtener una zona turística actual si es que el turista se encuentra en alguna zona turística registrada, para ello es necesario que se tenga habilitada la geolocalización del smartphone del turista. 	
	}
		\UCitem{Versión}{\color{Gray}1.0}
		\UCitem{Actor}{\cdtRef{actor:turista}{Turista}}
		\UCitem{Propósito}{Visualizar el listado de las zonas turísticas.}
		\UCitem{Entradas}{
			\begin{UClist}
				\UCli Zonas turísticas
				\UCli Posicionamiento
			\end{UClist}
		}
		\UCitem{Origen}{\ioSistema}
		\UCitem{Salidas}{Listado de Zonas Turísticas}
		\UCitem{Precondiciones}{Haber estado en alguna zona turística y tener habilitada la geolocalización}
		\UCitem{Postcondiciones}{Podrá visualizar las zonas turísticas.}
		\UCitem{Reglas de negocio}{Ninguna}
		\UCitem{Errores}{Ninguno}
		\UCitem{Tipo}{Caso de uso primario}
	\end{UseCase}
%--------------------------------------
	\begin{UCtrayectoria} 
		\UCpaso [\UCactor] Toca la notificación de área turística o ingresa a la aplicación después de iniciar sesión.
		
		\includeUC{PO-CU2.2}{Obtener Posicionamiento}
		
		\UCpaso Verifica a través del GPS que el usuario se encuentra dentro de un área turística registrada en el sistema. \refTray{A}.
		
		\UCpaso \label{LR-CU2:ZonaActual} Obtiene la zona turística actual en la que se encuentra el turista.
		
		\UCpaso \label{LR-CU2:ZonaVisitada} Obtiene las zonas turísticas visitadas por el turista.
		
		\UCpaso Establece el catálogo \textbf{Zona Turística Actual} con la información obtenida en el paso \ref{LR-CU2:ZonaActual} de la trayectoria principal.
		
		\UCpaso Establece el catálogo \textbf{Zonas Turísticas} con la información obtenida en el paso \ref{LR-CU2:ZonaVisitada} de la trayectoria principal.
		
		\UCpaso \label{LR-CU2:Pantalla} Muestra la pantalla \IUref{LR-IU2}{Zonas turísticas} con el listado de las zonas turísticas obtenidas.
		
	\end{UCtrayectoria}

%--------------------------------------		
		\begin{UCtrayectoriaA}{A}{El turista no se encuentra ubicado en alguna zona turística registrada}
			
			\UCpaso \label{LR-CU2:ZonaVisitadaA} Obtiene las zonas turísticas visitadas por el turista.
			
			\UCpaso Establece el catálogo \textbf{Zonas Turísticas} con la información obtenida en el paso \ref{LR-CU2:ZonaVisitadaA} de la trayectoria alterna \textbf{A}.
			
			\UCpaso Continúa en el paso \ref{LR-CU2:Pantalla} la trayectoria principal.
			
		\end{UCtrayectoriaA}
	
	