% \IUref{IUAdmPS}{Administrar Planta de Selección}
% \IUref{IUModPS}{Modificar Planta de Selección}
% \IUref{IUEliPS}{Eliminar Planta de Selección}

% 


% Copie este bloque por cada caso de uso:
%-------------------------------------- COMIENZA descripción del caso de uso.

%\begin{UseCase}[archivo de imágen]{UCX}{Nombre del Caso de uso}{
%--------------------------------------
	\begin{UseCase}{LR-CU2.1.1}{Generar Ruta Turística}{
		
		Permite llevar a cabo la generación de una ruta turística a partir de la ubicación actual del turista, para ello el turista debe encontrarse, mediante geolocaclización, dentro una zona turística para poder ser generada; de lo contrario no se podrá generar una ruta.\\
		
		Para poder generar la ruta, además de encontrarse dentro de la zona turística, el turista deberá seleccionar los parámetros que requiera para la generación de ruta, estos parámetros son: tipo de ruta, cantidad de servicios y ordenar por puntuación.\\
		
		Las rutas turísticas podrán variar dependiendo los parámetros seleccionados por el turista o por la zona en la que se encuentra.
	}
		\UCitem{Versión}{\color{Gray}1.0}
		\UCitem{Actor}{\cdtRef{actor:turista}{Turista}}
		\UCitem{Propósito}{Generar una ruta turística para el turista.}
		\UCitem{Entradas}{
			\begin{UClist}
				\UCli Zona turística
				\UCli Posicionamiento
				\UCli Tipo de ruta turística
				\UCli Cantidad de servicios
				\UCli Ordenamiento
			\end{UClist}
		}
		\UCitem{Origen}{
			\begin{UClist}
				\UCli \ioSistema
				\UCli \ioSeleccionar
			\end{UClist}
		}
		\UCitem{Salidas}{Mapa con la ruta turística}
		\UCitem{Precondiciones}{Encontrarse dentro de una zona turística.}
		\UCitem{Postcondiciones}{Podrá visualizar la ruta turística.}
		\UCitem{Reglas de negocio}{Ninguna}
		\UCitem{Errores}{\UCerr{Uno}{Cuando el turista no se encuentra en una zona turística registrada,}{muestra el mensaje \cdtIdRef{MSG}{No se puede generar ruta} y regresa al paso \ref{LR-CU2.1:Pantalla1} de la trayectoria principal.}}
		\UCitem{Tipo}{Extiende de \cdtIdRef{LR-CU2.1}{Visualizar Actividades Turísticas}}
	\end{UseCase}
%--------------------------------------
	\begin{UCtrayectoria} 
		\UCpaso [\UCactor] Solicita generar una ruta turística a partir de su ubicación actual seleccionando la opción \textbf{Generar ruta a partir de la ubicación actual} en la pantalla \IUref{LR-IU2.1a}{Zona turística}.
		
		\UCpaso Verifica que el turista se encuentre actualmente en una zona turística registrada. \refErr{Uno}
		
		\UCpaso Muestra la pantalla \IUref{LR-IU2.1.1a}{Seleccionar Parámetros} 
		
		\UCpaso [\UCactor] Ingresa los parámetros para la generación de la ruta.
		
		\UCpaso [\UCactor] Solicita continuar con la operación tocando el botón \IUbutton{Aceptar}
		
		\UCpaso Genera la ruta con base en los parámetros ingresados.
		
		\UCpaso Muestra la pantalla \IUref{LR-IU2.1.1b}{Ruta Turística} con la ruta turística trazada en el mapa.
		
	\end{UCtrayectoria}


	
	