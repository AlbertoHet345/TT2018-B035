% \IUref{IUAdmPS}{Administrar Planta de Selección}
% \IUref{IUModPS}{Modificar Planta de Selección}
% \IUref{IUEliPS}{Eliminar Planta de Selección}

% 


% Copie este bloque por cada caso de uso:
%-------------------------------------- COMIENZA descripción del caso de uso.

%\begin{UseCase}[archivo de imágen]{UCX}{Nombre del Caso de uso}{
%--------------------------------------
	\begin{UseCase}{LR-CU2.1.1}{Generar Ruta Turística}{
		
		Permite llevar a cabo la generación de una ruta turística a partir de la ubicación actual del turista, para ello el turista debe encontrarse, mediante geolocaclización, dentro una zona turística para poder ser generada; de lo contrario no se podrá generar una ruta.\\
		
		Para poder generar la ruta, además de encontrarse dentro de la zona turística, el turista deberá seleccionar los parámetros que requiera para la generación de ruta, estos parámetros son: tipo de ruta, cantidad de servicios y ordenar por puntuación.\\
		
		Las rutas turísticas podrán variar dependiendo los parámetros seleccionados por el turista o por la zona en la que se encuentra.
	}
		\UCitem{Versión}{\color{Gray}1.0}
		\UCitem{Actor}{\cdtRef{actor:turista}{Turista}}
		\UCitem{Propósito}{Generar una ruta turística para el turista.}
		\UCitem{Entradas}{
			\begin{UClist}
				\UCli Zona turística
				\UCli Posicionamiento
				\UCli Cantidad de servicios
			\end{UClist}
		}
		\UCitem{Origen}{
			\begin{UClist}
				\UCli \ioSistema
				\UCli \ioSeleccionar
			\end{UClist}
		}
		\UCitem{Salidas}{Mapa con la ruta turística}
		\UCitem{Precondiciones}{Encontrarse dentro de una zona turística.}
		\UCitem{Postcondiciones}{Podrá visualizar la ruta turística.}
		\UCitem{Reglas de negocio}{\cdtIdRef{BR-002}{Localización para generación de rutas}}
		\UCitem{Errores}{\UCerr{Uno}{Cuando no se puede llevar a cabo la operación de manera correca,}{se muestra el mensaje \cdtIdRef{MSG6}{Operación fallida} y termina el caso de uso.}}
		\UCitem{Tipo}{Extiende de \cdtIdRef{LR-CU2.1}{Visualizar Actividades Turísticas}}
	\end{UseCase}
%--------------------------------------
	\begin{UCtrayectoria} 
		\UCpaso [\UCactor] Solicita generar una ruta turística tocando el botón \IUbutton{Ver mapa} en la pantalla \IUref{LR-IU2.1a}{Zona Turística}.
		
		\UCpaso Muestra la pantalla \IUref{LR-IU2.1b}{Mapa de Zona Turística}.
		
		\UCpaso Verifica que el actor se encuentre dentro de la zona turística para generar la ruta, con base en la regla de negocio \cdtIdRef{BR-002}{Localización para generación de rutas}. \refTray{A}
		
		\UCpaso Hablita el botón \IUGenerarRuta{}.
		
		\UCpaso [\UCactor] Solicita generar una ruta turística tocando el botón \IUGenerarRuta{}. 
		
		\UCpaso Obtiene el nombre y la ubicación de los servicios turísticos que pertenecen a la zona turística.
		
		\UCpaso Muestra la pantalla \IUref{LR-IU2.1.1a}{Generar Ruta Selección de Servicios Turísticos}.
		
		\UCpaso [\UCactor] Selecciona los servicios turísticos para generar su ruta.
		
		\UCpaso [\UCactor] Solicita finalizar la operación tocando el botón \IUbutton{Seleccionar}. \refTray{B}
		
		\UCpaso Genera la ruta turística tomando en cuenta el orden en que fueron seleccionados los servicios turísticos. \refErr{Uno}
		
		\UCpaso Muestra la pantalla \IUref{LR-IU2.1.1b}{Generar Ruta} con la ruta trazada en el mapa.
		
		\UCpaso Habilita el botón \IUbutton{Comenzar Ruta}.
		
		\UCpaso [\UCactor] Solicita comenzar la navegación tocando el botón \IUbutton{Comenzar Ruta}. \refTray{C}
		
		\UCpaso Comienza la navegación en el mapa.
		
	\end{UCtrayectoria}

	\begin{UCtrayectoriaA}[Fin del caso de uso]{A}{Cuando el actor no se encuentra en la zona turística.}
		
		\UCpaso Inhabilita el botón \IUGenerarRuta{}.
		
		\UCpaso Muestra la pantalla \IUref{LR-IU2.1b}{Mapa de Zona Turística}.
		
	\end{UCtrayectoriaA}

	\begin{UCtrayectoriaA}[Fin del caso de uso]{B}{Cuando el actor requiere cancelar la operación.}
		
		\UCpaso [\UCactor] Solicita cancelar la operación tocando el botón \IUbutton{Cancelar}.
		
		\UCpaso Regresa a la pantalla anterior.
		
	\end{UCtrayectoriaA}

	\begin{UCtrayectoriaA}[Fin del caso de uso]{C}{Cuando el actor no quiere continuar con su ruta turística.}
		
		\UCpaso [\UCactor] Solicita cancelar su ruta turística tocando el botón \IUbutton{< Atrás}
		
		\UCpaso Elimina la ruta turistica.
		
		\UCpaso Regresa a la pantalla \IUref{LR-IU2.1}{Visualizar Información de Zona Turística}
		
	\end{UCtrayectoriaA}
	
	