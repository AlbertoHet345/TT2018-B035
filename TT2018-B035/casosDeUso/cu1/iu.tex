\newpage
\section{RE-IU1 Registrar Usuario}

\subsection{Objetivo}
Permite el registro de un nuevo turista en el sistema. Es importante el registro para que se pueda utilizar la aplicación móvil y con ella se pueda tener acceso al sistema. El registro puede hacerse mediante Facebook, Google o por correo electrónico. Así mismo en esta pantalla se podrá iniciar sesión, si es que el usuario ya tiene creada su cuenta.

\subsection{Diseño}
La Figura \ref{RE-IU1a} muestra la pantalla \IUref{RE-IU1a}{Registrar Usuario} que es la pantalla principal que mostrará la aplicación cuando un usuario no se encuentra registrado aún, en ella se puede observar cuatro botones, cada uno de estos tiene una acción diferente. Los botones \textbf{Entrar con Facebook} y \textbf{Entrar con Google} permiten al usuario registrarse, así como iniciar sesión, si se selecciona alguno de estos botones se mostrará la figura \ref{RE-IU1b} con la pantalla \IUref{RE-IU1b}{Registrar Usuario por Facebook/Google}. El botón \textbf{Iniciar Sesión} permite al usuario que ya se encuentra registrado acceder a la aplicación mediante el llenado del formulario de la figura \ref{RE-IU1c} con la pantalla \IUref{RE-IU1c}{Iniciar Sesión}. Por último, el botón \textbf{Registrarse} permite al usuario darse de alta en la aplicación ingresando la información correspondiente en la figura \ref{RE-IU1d} pantalla \IUref{RE-IU1d}{Registrar Usuario por correo}.

\IUfig[.2]{../CasosDeUso/cu1/images/RE-IU1a}{RE-IU1a}{Registrar Usuario}

\IUfig[.2]{../CasosDeUso/cu1/images/RE-IU1b}{RE-IU1b}{Registrar Usuario por Facebook/Google}

\IUfig[.2]{../CasosDeUso/cu1/images/RE-IU1c}{RE-IU1c}{Iniciar Sesión}

\IUfig[.2]{../CasosDeUso/cu1/images/RE-IU1d}{RE-IU1d}{Registrar Usuario por correo}

