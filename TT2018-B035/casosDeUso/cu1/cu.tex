% \IUref{IUAdmPS}{Administrar Planta de Selección}
% \IUref{IUModPS}{Modificar Planta de Selección}
% \IUref{IUEliPS}{Eliminar Planta de Selección}

% 


% Copie este bloque por cada caso de uso:
%-------------------------------------- COMIENZA descripción del caso de uso.

%\begin{UseCase}[archivo de imágen]{UCX}{Nombre del Caso de uso}{
%--------------------------------------
	\begin{UseCase}{RE-CU1}{Registrar Usuario}{
		Permite al \hyperlink{actor:turista}{Turista} ingresar su información básica para poder tener una cuenta única dentro del sistema. Es importante el registro en la aplicación, ya que de esta forma el turista podrá acceder a la información de las zonas turísticas que se encuentran en el sistema.\\
		
		El registro de un nuevo usuario sólo podrá realizarse mediante Inicio de Sesión con Apple (Sign in with Apple\footnote{Ver \url{https://developer.apple.com/sign-in-with-apple/}}, por su nombre en inglés) debido a que la aplicación es únicamente para dispositivos iPhone y permitirá que la aplicación cuente sólo con usuarios reales ya que para ingresar deberá contar con un Apple ID.\\
		
		Así mismo este caso de uso permitirá el inicio de sesión, en caso de que el turista ya se encuentre registrado en la aplicación.
	}
		\UCitem{Versión}{\color{Gray}1.0}
		\UCitem{Actor}{\hyperlink{actor:turista}{Turista}}
		\UCitem{Propósito}{Que el turista pueda registrarse para acceder a la aplicación o, si ya se encuentra registrado, iniciar sesión en la misma.}
		\UCitem{Entradas}{
			\begin{UClist}
				\UCli Apple ID del turista.
				\UCli Correo electrónico asociado al Apple ID del turista.
				\UCli Nombre del turista asociado al Apple ID.
			\end{UClist}
		}
		\UCitem{Origen}{
			\begin{UClist}
				\UCli \ioSistema
				\UCli \ioEscribir
			\end{UClist}
		}
		\UCitem{Salidas}{Ninguna}
		\UCitem{Precondiciones}{
			\begin{UClist}
				\UCli El turista debe haber instalado la aplicación en su iPhone.
				\UCli El turista debe contar con un Apple ID.
				\UCli El turista debe tener una cuenta creada en el sistema para iniciar sesión.
			\end{UClist}
		}
		\UCitem{Postcondiciones}{El usuario quedará registrado en el sistema, con una cuenta única, o accederá a la aplicación.}
		\UCitem{Reglas de negocio}{\cdtIdRef{BR-001}{Iniciar Sesión con Apple ID}}
		\UCitem{Errores}{
%			\begin{UClist}
%				\UCli \UCerr{Uno}{Cuando el correo ingresado no el correo ingresado ya se encuentra registrado en el sistema,}{se muestra el mensaje \cdtIdRef{MSG3}{Correo existente} y continua en el paso \ref{RE-CU1:IntroduceInformacion} de la trayectoria principal.}
%				\UCli \UCerr{Dos}{Cuando el correo ingresado es nulo o no cumple con el tipo de dato y longitud,}{se muestra el mensaje \cdtIdRef{MSG2}{Correo inválido} y continua en el paso \ref{RE-CU1:IntroduceInformacion} de la trayectoria principal.}
%				\UCli \UCerr{Tres}{Cuando la contraseña ingresada no cumple con la regla \cdtIdRef{BR-001}{Longitud válida para contraseñas},}{se muestra el mensaje \cdtIdRef{MSG4}{Formato de campo inválido} y continua en el paso \ref{RE-CU1:IntroduceInformacion} de la trayectoria principal.}
%				\UCli \UCerr{Cuatro}{Cuando el nombre de usuario ingresado no cumple la regla \cdtIdRef{BR-002}{Longitud válida para nombre de usuario},}{muestra el mensaje \cdtIdRef{MSG4}{Formato de campo inválido} y continua \ref{RE-CU1:IntroduceInformacion} de la trayectoria principal.}
%				\UCli \UCerr{Cinco}{Cuando el turista no acepta los términos y condiciones,}{muestra el mensaje \cdtIdRef{MSG5}{Campo Obligatorio} y continua en el paso \ref{RE-CU1:IntroduceInformacion} de la trayectoria principal.}
%				\UCli \UCerr{Seis}{Cuando no se puede llevar a cabo la operación de manera exitosa,}{muestra el mensaje \cdtIdRef{MSG6}{Operación fallida} y termina el caso de uso.}
%				\UCli \UCerr{Siete}{Cuando el correo electrónico o la contraseña ingresados son incorrectos,}{muestra el mensaje \cdtIdRef{MSG7}{Correo y/o contraseña incorrectos} y continua en el paso \ref{RE-CU1:iniciarSesion} de la trayectoria alterna \textbf{A}.}
%			\end{UClist}
		}
		\UCitem{Tipo}{Caso de uso primario}
	\end{UseCase}
%--------------------------------------
	\begin{UCtrayectoria}
		\UCpaso [\UCactor] Ingresa a la aplicación tocando el icono \IUref{IU0}{Icono de la aplicación} en la pantalla de su teléfono.
		
		\UCpaso Se mostrará la pantalla \IUref{RE-IU1a}{Login}, la cual contiene el botón para iniciar sesión mediante Apple ID.
		
		\UCpaso [\UCactor] Solicita registrarse en la aplicación tocando el botón \IUBotonApple{}.
		
		\UCpaso Verifica si el usuario se encuentra registrado en la aplicación. \refTray{A}
		
		\UCpaso Se mostrará la pantalla \IUref{RE-IU1b}{Iniciar Sesión con Apple} la cual muestra el anuncio de privacidad emitido por Apple.
		
		\UCpaso [\UCactor] Solicita continuar con el registro tocando el botón \IUbutton{Continuar}. \refTray{B}
		
		\UCpaso Se mostrará la pantalla \IUref{RE-IU1c}{Apple ID}, la cual muestra la información que el usuario podrá elegir mostrar o no.
		
		\UCpaso [\UCactor] \label{RE-CU1:Correo} Solicita mostrar su correo electrónico seleccionando la opción \textbf{Compartir mi correo}. \refTray{C} \refTray{D}
		
		\UCpaso \label{RE-CU1:Boton} Habilita el botón \IUbutton{Continuar con mi contraseña}.
		
		\UCpaso [\UCactor] Solicita continuar con la operación tocando el botón \IUbutton{Continuar con mi contraseña}. \refTray{B}
		
		\UCpaso Se mostrará la pantalla \IUref{RE-IU1d}{Apple ID Contraseña}.
		
		\UCpaso [\UCactor] Ingresa la contraseña de su Apple ID.
		
		\UCpaso [\UCactor] Solicita continuar con el registro tocando el botón \IUbutton{Continuar}. \refTray{E}
		
		\UCpaso Verifica que la contraseña ingresada pertenezca al Apple ID del usuario. \refErr{Uno}
		
		\UCpaso Realiza el registro del usuario. \refErr{Dos}
		
		\UCpaso \label{RE-CU1:PantallaZonasTuristicas} Muestra la pantalla \IUref{LR-IU2}{Zonas Turísticas}.
	
	\end{UCtrayectoria}

%--------------------------------------		
	\begin{UCtrayectoriaA}{A}{Cuando el turista ya se encuentra registrado en el sistema.}
			
			\UCpaso Muestra la pantalla \IUref{RE-IU1d}{Apple ID Contraseña}.
			
			\UCpaso [\UCactor] Ingresa la contraseña de su Apple ID.
			
			\UCpaso [\UCactor] Solicita continuar con la operación tocando el botón \IUbutton{Continuar}. \refTray{B}
			
			\UCpaso Verifica que la contraseña ingresada pertenezca al Apple ID del usuario. \refErr{Uno}
			
			\UCpaso Continua en el paso \ref{RE-CU1:PantallaZonasTuristicas} de la trayectoria principal.
			
	\end{UCtrayectoriaA}
		
%--------------------------------------
	\begin{UCtrayectoriaA}[Fin del caso de uso]{B}{Cuando el actor requiere cancelar la operación}
		
		\UCpaso [\UCactor] Solicita cancelar la operación tocando el botón \IUbutton{Cancelar}.
		
		\UCpaso Regresa a la pantalla \IUref{RE-IU1a}{Login}.
		
	\end{UCtrayectoriaA}
	
	\begin{UCtrayectoriaA}{C}{Cuando el actor requiere cambiar su nombre de usuario.}
		
		\UCpaso [\UCactor] Solicita cambiar su nombre de usuario tocando el icono \IUCambiarNombre{} en la pantalla \IUref{RE-IU1c}{Apple ID}.
		
		\UCpaso Se mostrará la pantalla \IUref{RE-IU1e}{Cambiar Nombre}.
		
		\UCpaso [\UCactor] Ingresa los datos solicitados.
		
		\UCpaso [\UCactor] Solicita finalizar la edición tocando el botón \IUbutton{Listo}.
		
		\UCpaso Continua en el paso \ref{RE-CU1:Correo} de la trayectoria principal.
		
	\end{UCtrayectoriaA}

	\begin{UCtrayectoriaA}{D}{Cuando el actor requiere ocultar su correo electrónico.}
		
		\UCpaso [\UCactor] Solicita ocultar su correo electrónico tocando la opción \textbf{Ocultar mi correo} en la pantalla \IUref{RE-IU1c}{Apple ID}.
		
		\UCpaso Continua en el paso \ref{RE-CU1:Boton} de la trayectoria principal.
		
	\end{UCtrayectoriaA}

%--------------------------------------
