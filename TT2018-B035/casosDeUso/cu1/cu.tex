% \IUref{IUAdmPS}{Administrar Planta de Selección}
% \IUref{IUModPS}{Modificar Planta de Selección}
% \IUref{IUEliPS}{Eliminar Planta de Selección}

% 


% Copie este bloque por cada caso de uso:
%-------------------------------------- COMIENZA descripción del caso de uso.

%\begin{UseCase}[archivo de imágen]{UCX}{Nombre del Caso de uso}{
%--------------------------------------
	\begin{UseCase}{CU1}{Registrar Usuario}{
		Este caso de uso permite al usuario ingresar su información básica, esto para poder tener una cuenta única dentro del sistema y estar registrado con su información asociada. 
	}
		\UCitem{Versión}{\color{Gray}1.0}
		\UCitem{Actor}{\hyperlink{Usuario}{Usuario}}
		\UCitem{Propósito}{Que el usuario pueda registrarse para acceder al sistema.}
		\UCitem{Entradas}{Correo electrónico, Contraseña, Nombre de Usuario.}
		\UCitem{Origen}{Teclado}
		\UCitem{Salidas}{N.A.}
		\UCitem{Precondiciones}{El usurario debe haber instalado la aplicación en su teléfono}
		\UCitem{Postcondiciones}{El usuario quedará registrado en el sistema, con una cuenta única.}
		\UCitem{Errores}{}
		\UCitem{Tipo}{Caso de uso primario}
		\UCitem{Observaciones}{}
	\end{UseCase}
%--------------------------------------
	\begin{UCtrayectoria}
		\UCpaso Despliega la pantalla  \IUref{IU1}{Bienvenida}.
		\UCpaso[\UCactor] Da clic en el botón continuar de la pantalla \IUref{IU1}{Bienvenida}.
		\UCpaso Despliega la pantalla \IUref{IU2}{Login} mostrando las diferentes opciones de registro \IUbutton{Entrar con FB}, \IUbutton{Entrar con Google}, \IUbutton{Iniciar Sesión} y \IUbutton{Registrarse}.
		\UCpaso[\UCactor] Selecciona registrarse mediante correo electrónico, dando clic en la opción \IUbutton{Registrarse}.\Trayref{A}.\Trayref{B}.
		\UCpaso Despliega la pantalla \IUref{IU4}{Registro por Correo} mostrando los campos que el usuario debe ingresar.
		\UCpaso[\UCactor] Introduce y selecciona la información solicitada en pantalla, posteriormente da clic en el botón \IUbutton{Registrarse}.\label{CU1-Registrar}.
		\UCpaso Valida que el correo electrónico no se haya registrado anteriormente y no sea nulo.\Trayref{C}.
		\UCpaso Valida que la contraseña cumpla con la longitud de 8 a 10 caracteres y que esta cuente con números y letras.\Trayref{D}.	
		\UCpaso Valida que el nombre de usuario cumpla con la longitud de 2 a 30 caracteres y unicamente se ingresen letras.\Trayref{E}.
		\UCpaso Valida que se haya seleccionado la opción de aceptar términos y condiciones.\Trayref{F}.
		\UCpaso Registra al usuario en el sistema con una cuenta única.
		\UCpaso Ingresa al sistema con la cuenta asociada.
		\UCpaso Termina el caso de uso.
		
	\end{UCtrayectoria}

%--------------------------------------		
		\begin{UCtrayectoriaA}{A}{El usuario selecciona iniciar sesión mediante correo electrónico, dando clic en la opción \IUbutton{Iniciar sesión}}
			
			\UCpaso Despliega la pantalla \IUref{IU5}{Iniciar Sesión} mostrando los campos que el usuario debe ingresar.
			
			\UCpaso[\UCactor] Introduce y selecciona la información solicitada en pantalla, posteriormente da clic en el botón \IUbutton{Ingresar}.\label{CU1-iniciar}.
			
			\UCpaso Valida que el correo electrónico se encuentre registrado y no sea nulo.\Trayref{A1}.

			\UCpaso Valida que la contraseña coincida con la asociada al correo electrónico previamente ingresado, y que respete la longitud de 8 a 10 caracteres.\Trayref{A2}.
			
			\UCpaso Ingresa al sistema con la cuenta asociada.
			
			\UCpaso Termina el caso de uso.
			
		\end{UCtrayectoriaA}
	
		\begin{UCtrayectoriaA}{A1}{El correo electrónico no se encuentra registrado o es nulo.}
		
		\UCpaso Muestra el mensaje arriba del campo correo electrónico{\bf MSG1 -}``Correo  inválido''.
		
		\UCpaso Continúa en el paso \ref{CU1-Registrar} del \UCref{CU1}.
		
	\end{UCtrayectoriaA}

		\begin{UCtrayectoriaA}{A2}{La contraseña no coincide con la asociada a la cuenta o no corresponde a la longitud del campo.}
		
		\UCpaso Muestra el mensaje arriba del campo Contraseña {\bf MSG1 -}``Contraseña inválida''.
		
		\UCpaso Continúa en el paso \ref{CU1-Registrar} del \UCref{CU1}.
		
	\end{UCtrayectoriaA}
		
%--------------------------------------
		\begin{UCtrayectoriaA}{B}{El usuario selecciona iniciar sesión mediante Facebook o una cuenta de Google, dando clic en la opción \IUbutton{Entrar con FB}, \IUbutton{Entrar con Google}}
			
			\UCpaso Verifica la cuenta de facebook o google asociada en el sistema del teléfono.\Trayref{B1}.
				
			\UCpaso Despliega la pantalla emergente \IUref{IU6}{Iniciar con FB,G} mostrando la cuenta asociada al teléfono inteligente ya sea de Facebook o Google respectivamente.
			
			\UCpaso[\UCactor] Acepta ingresar con la cuenta mostrada dando clic en el botón  \IUbutton{Aceptar}.
			\UCpaso Ingresa al sistema con la cuenta asociada.
		
			\UCpaso Termina el caso de uso.
			
		\end{UCtrayectoriaA}
	
		\begin{UCtrayectoriaA}{B1}{El usuario no mantiene una cuenta de Facebook o Google abierta en el sistema del teléfono.}
		
		\UCpaso Redirige a la aplicación correspondiente, ya sea Facebook o Google para su registro.
		
		\UCpaso Termina el caso de uso.
		
	\end{UCtrayectoriaA}

%--------------------------------------
		\begin{UCtrayectoriaA}{C}{El correo electrónico ya esta registrado o es nulo.}
			
			\UCpaso Muestra el mensaje arriba del campo correo electrónico{\bf MSG1 -}``Correo Electrónico inválido''.
			
			\UCpaso Continúa en el paso \ref{CU1-Registrar} del \UCref{CU1}.
			
		\end{UCtrayectoriaA}

%--------------------------------------
		\begin{UCtrayectoriaA}{D}{La contraseña no coincide con la longitud o el tipo de dato.}
			
			\UCpaso Muestra el mensaje arriba del campo Contraseña {\bf MSG1 -}``Contraseña inválida''.
				
			\UCpaso Continúa en el paso \ref{CU1-Registrar} del \UCref{CU1}.
		
	\end{UCtrayectoriaA}

	\begin{UCtrayectoriaA}{E}{El nombre no cumple con la longitud o no corresponde el tipo de dato.}
	
		\UCpaso Muestra el mensaje arriba del campo Nombre de Usuario {\bf MSG1 -}``Nombre inválido''.
	
		\UCpaso Continúa en el paso \ref{CU1-Registrar} del \UCref{CU1}.
	
\end{UCtrayectoriaA}

	\begin{UCtrayectoriaA}{F}{No se aceptaron los términos y condiciones.}
	
		\UCpaso Muestra el mensaje arriba del campo Términos y Condiciones {\bf MSG1 -}``Requerido''.
	
	\UCpaso Continúa en el paso \ref{CU1-Registrar} del \UCref{CU1}.
	
	\end{UCtrayectoriaA}

%--------------------------------------

		
		