% \IUref{IUAdmPS}{Administrar Planta de Selección}
% \IUref{IUModPS}{Modificar Planta de Selección}
% \IUref{IUEliPS}{Eliminar Planta de Selección}

% 


% Copie este bloque por cada caso de uso:
%-------------------------------------- COMIENZA descripción del caso de uso.

%\begin{UseCase}[archivo de imágen]{UCX}{Nombre del Caso de uso}{
%--------------------------------------
	\begin{UseCase}{RE-CU1}{Registrar Usuario}{
		Permite al \hyperlink{actor:turista}{Turista} ingresar su información básica para poder tener una cuenta única dentro del sistema. Es importante el registro en la aplicación, ya que de esta forma el turista podrá acceder a la información de las zonas turísticas que se encuentran en el sistema.\\
		
		El registro de un nuevo usuario sólo podrá realizarse mediante Inicio de Sesión con Apple (Sign in with Apple\footnote{Ver \url{https://developer.apple.com/sign-in-with-apple/}}, por su nombre en inglés) debido a que la aplicación es únicamente para dispositivos iPhone y permitirá que la aplicación cuente sólo con usuarios reales ya que para ingresar deberá contar con un Apple ID.\\
		
		Así mismo este caso de uso permitirá el inicio de sesión, en caso de que el turista ya se encuentre registrado en la aplicación.
	}
		\UCitem{Versión}{\color{Gray}1.0}
		\UCitem{Actor}{\hyperlink{actor:turista}{Turista}}
		\UCitem{Propósito}{Que el turista pueda registrarse para acceder a la aplicación o, si ya se encuentra registrado, iniciar sesión en la misma.}
		\UCitem{Entradas}{
			\begin{UClist}
				\UCli Apple ID del turista.
				\UCli Correo electrónico asociado al Apple ID del turista.
				\UCli Nombre del turista asociado al Apple ID.
			\end{UClist}
		}
		\UCitem{Origen}{
			\begin{UClist}
				\UCli \ioSistema
				\UCli \ioEscribir
			\end{UClist}
		}
		\UCitem{Salidas}{Ninguna}
		\UCitem{Precondiciones}{
			\begin{UClist}
				\UCli El turista debe haber instalado la aplicación en su iPhone.
				\UCli El turista debe contar con un Apple ID.
				\UCli El turista debe tener una cuenta creada en el sistema para iniciar sesión.
			\end{UClist}
		}
		\UCitem{Postcondiciones}{El usuario quedará registrado en el sistema, con una cuenta única, o accederá a la aplicación.}
		\UCitem{Reglas de negocio}{\cdtIdRef{BR-001}{Iniciar Sesión con Apple}}
		\UCitem{Errores}{
%			\begin{UClist}
%				\UCli \UCerr{Uno}{Cuando el correo ingresado no el correo ingresado ya se encuentra registrado en el sistema,}{se muestra el mensaje \cdtIdRef{MSG3}{Correo existente} y continua en el paso \ref{RE-CU1:IntroduceInformacion} de la trayectoria principal.}
%				\UCli \UCerr{Dos}{Cuando el correo ingresado es nulo o no cumple con el tipo de dato y longitud,}{se muestra el mensaje \cdtIdRef{MSG2}{Correo inválido} y continua en el paso \ref{RE-CU1:IntroduceInformacion} de la trayectoria principal.}
%				\UCli \UCerr{Tres}{Cuando la contraseña ingresada no cumple con la regla \cdtIdRef{BR-001}{Longitud válida para contraseñas},}{se muestra el mensaje \cdtIdRef{MSG4}{Formato de campo inválido} y continua en el paso \ref{RE-CU1:IntroduceInformacion} de la trayectoria principal.}
%				\UCli \UCerr{Cuatro}{Cuando el nombre de usuario ingresado no cumple la regla \cdtIdRef{BR-002}{Longitud válida para nombre de usuario},}{muestra el mensaje \cdtIdRef{MSG4}{Formato de campo inválido} y continua \ref{RE-CU1:IntroduceInformacion} de la trayectoria principal.}
%				\UCli \UCerr{Cinco}{Cuando el turista no acepta los términos y condiciones,}{muestra el mensaje \cdtIdRef{MSG5}{Campo Obligatorio} y continua en el paso \ref{RE-CU1:IntroduceInformacion} de la trayectoria principal.}
%				\UCli \UCerr{Seis}{Cuando no se puede llevar a cabo la operación de manera exitosa,}{muestra el mensaje \cdtIdRef{MSG6}{Operación fallida} y termina el caso de uso.}
%				\UCli \UCerr{Siete}{Cuando el correo electrónico o la contraseña ingresados son incorrectos,}{muestra el mensaje \cdtIdRef{MSG7}{Correo y/o contraseña incorrectos} y continua en el paso \ref{RE-CU1:iniciarSesion} de la trayectoria alterna \textbf{A}.}
%			\end{UClist}
		}
		\UCitem{Tipo}{Caso de uso primario}
	\end{UseCase}
%--------------------------------------
	\begin{UCtrayectoria}
		\UCpaso [\UCactor] Ingresa a la aplicación tocando el icono \IUref{IU0}{Icono de la aplicación} en la pantalla de su teléfono.
		
		\UCpaso Despliega la pantalla  \IUref{IU1}{Bienvenida}.
		
		\UCpaso [\UCactor] Continua tocando el  botón \IUbutton{Continuar} de la pantalla \IUref{IU1}{Bienvenida}.
		
		\UCpaso Muestra la pantalla \IUref{RE-IU1a}{Registrar Usuario} mostrando las diferentes opciones de registro \IUbutton{Entrar con FB}, \IUbutton{Entrar con Google}, \IUbutton{Iniciar Sesión} y \IUbutton{Registrarse}.
		
		\UCpaso [\UCactor] Selecciona registrarse mediante correo electrónico, tocando la opción \IUbutton{Registrarse}. \refTray{A} \refTray{B}.
		
		\UCpaso Muestra la pantalla \IUref{RE-IU1d}{Registrar Usuario por Correo} mostrando los campos que el usuario debe ingresar.
		
		\UCpaso [\UCactor] \label{RE-CU1:IntroduceInformacion}. Introduce y selecciona la información solicitada en pantalla.
		
		\UCpaso [\UCactor] Solicita finalizar el registro tocando el botón \IUbutton{Registrarse}.
		
		\UCpaso Verifica que el correo electrónico ingresado no se haya registrado anteriormente y no sea nulo.\refErr{Uno} \refErr{Dos} \refTray{C}
		
		\UCpaso Verifica que la contraseña cumpla con la longitud y el tipo de dato, con base en la regla \cdtIdRef{BR-001}{Longitud válida para contraseñas}. \refErr{Tres} \refTray{D}.
			
		\UCpaso Verifica que el nombre de usuario cumpla con la longitud y el tipo de dato, con base en la regla \cdtIdRef{BR-002}{Longitud válida para nombre de usuario}. \refErr{Cuatro} \refTray{E}.
		
		\UCpaso Verifica que se haya seleccionado la opción de aceptar términos y condiciones. \refErr{Cinco} \refTray{F}.
		
		\UCpaso Registra al usuario en el sistema con una cuenta única. 
		
		\UCpaso Muestra el mensaje \cdtIdRef{MSG1}{Operación exitosa} indicando que el registro fue exitoso.
		
		\UCpaso \label{RE-CU1:Pantalla}Muestra la pantalla \IUref{LR-IU2}{Zonas Turísticas}.
		
	\end{UCtrayectoria}

%--------------------------------------		
	\begin{UCtrayectoriaA}{A}{El turista requiere iniciar sesión mediante correo electrónico, tocando la opción \IUbutton{Iniciar sesión}}
			
		\UCpaso Despliega la pantalla \IUref{IU5}{Iniciar Sesión} mostrando los campos que el usuario debe ingresar.
			
		\UCpaso [\UCactor] \label{RE-CU1:iniciarSesion}. Introduce y selecciona la información solicitada en pantalla.
		
		\UCpaso [\UCactor] Solicita iniciar sesión tocando el botón \IUbutton{Ingresar}. 
			
		\UCpaso Verifica que el correo electrónico se encuentre registrado y no sea nulo. \refErr{Siete} 

		\UCpaso Verifica que la contraseña coincida con la asociada al correo electrónico ingresado. \refErr{Siete}
			
		\UCpaso Ingresa al sistema con la cuenta asociada.
			
		\UCpaso Continua en el paso \ref{RE-CU1:Pantalla} de la trayectoria principal.
			
	\end{UCtrayectoriaA}
		
%--------------------------------------
	\begin{UCtrayectoriaA}{B}{El turista selecciona iniciar sesión mediante Facebook o una cuenta de Google, tocando la opción \IUbutton{Entrar con FB} o \IUbutton{Entrar con Google}}
		
		\UCpaso Verifica la cuenta de Facebook o Google asociada en el sistema del teléfono. \refTray{C}.
				
		\UCpaso \label{RE-CU1:FB} Despliega la pantalla \IUref{RE-IU1b}{Registrar Usuario por Facebook/Google} mostrando la cuenta asociada al teléfono inteligente ya sea de Facebook o Google respectivamente.
			
		\UCpaso [\UCactor] Acepta ingresar con la cuenta mostrada tocando el botón  \IUbutton{Aceptar}.
		
		\UCpaso Ingresa al sistema con la cuenta asociada. \refErr{Seis}
		
		\UCpaso Continua en el paso \ref{RE-CU1:Pantalla} de la trayectoria principal.
		
	\end{UCtrayectoriaA}
	
	\begin{UCtrayectoriaA}{C}{El turista no mantiene una cuenta de Facebook o Google abierta en el sistema del teléfono.}
		
		\UCpaso Redirige a la aplicación correspondiente, ya sea Facebook o Google para su registro.
		
		\UCpaso [\UCactor] Realiza las acciones pertinentes para poder iniciar sesión.
		
		\UCpaso Continua en el paso \ref{RE-CU1:FB} de la trayectoria alterna \textbf{B}
		
	\end{UCtrayectoriaA}

%--------------------------------------

\subsection{Puntos de extensión}
\UCExtenssionPoint{Cuando el turista requiere editar su perfil}{En el paso \ref{RE-CU1:Pantalla} de la trayectoria principal.}{\cdtIdRef{RE-CU1.1}{Editar Perfil}}