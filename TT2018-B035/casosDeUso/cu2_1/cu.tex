% \IUref{IUAdmPS}{Administrar Planta de Selección}
% \IUref{IUModPS}{Modificar Planta de Selección}
% \IUref{IUEliPS}{Eliminar Planta de Selección}

% 


% Copie este bloque por cada caso de uso:
%-------------------------------------- COMIENZA descripción del caso de uso.

%\begin{UseCase}[archivo de imágen]{UCX}{Nombre del Caso de uso}{
%--------------------------------------
	\begin{UseCase}{LR-CU2.1}{Visualizar Actividades Turísticas}{
		
		Permite realizar la consulta de las actividades turísticas que pueden ser realizadas en una zona turística, estas actividades pueden  variar dependiendo de la zona en la que se encuentre el turista. La zona turística puede ser la zona actual o alguna otra que haya sido visitada por el turista.\\
		
	}
		\UCitem{Versión}{\color{Gray}1.0}
		\UCitem{Actor}{\cdtRef{actor:turista}{Turista}}
		\UCitem{Propósito}{Visualizar el listado de las zonas turísticas.}
		\UCitem{Entradas}{
			\begin{UClist}
				\UCli Zonas turísticas
				\UCli Tipo de actividades turísticas
			\end{UClist}
		}
		\UCitem{Origen}{\ioSistema}
		\UCitem{Salidas}{Tipos de actividades turísticas}
		\UCitem{Precondiciones}{Haber seleccionado una zona turística.}
		\UCitem{Postcondiciones}{Podrá visualizar las actividades turísticas.}
		\UCitem{Reglas de negocio}{Ninguna}
		\UCitem{Errores}{\UCerr{Uno}{Cuando el turista no se encuentra en una zona turística registrada,}{muestra el mensaje \cdtIdRef{MSG}{No se puede generar ruta} y regresa al paso \ref{LR-CU2.1:Pantalla1} de la trayectoria principal.}}
		\UCitem{Tipo}{Extiende de \cdtIdRef{LR-CU2}{Obtener Zonas Turísticas}}
	\end{UseCase}
%--------------------------------------
	\begin{UCtrayectoria} 
		\UCpaso [\UCactor] Solicita visualizar las actividades turísticas seleccionando una zona turística en la pantalla \IUref{LR-IU2}{Zonas turísticas}.
		
		\UCpaso Verifica que la zona turística seleccionada sea la actual. \refTray{A}
		
		\UCpaso Habilita las opciones \textbf{Generar ruta a partir de la ubicación actual} y \textbf{Servicios turísticos cerca de mí}.
		
		\UCpaso \label{LR-CU2.1:Pantalla1} Muestra la pantalla \IUref{LR-IU2.1a}{Zona turística}.
		
		\UCpaso [\UCactor] Solicita visualizar las actividades turísticas seleccionando la opción \textbf{Servicios turísticos cerca de mi}. \refTray{B}
		
		\UCpaso Obtiene los servicios turísticos asociados a la zona turística previamente seleccionada.
		
		\UCpaso \label{LR-CU2.1:Pantalla} Muestra la pantalla \IUref{LR-IU2.1b}{Servicios turísticos}.
		
	\end{UCtrayectoria}

%--------------------------------------		
		\begin{UCtrayectoriaA}{A}{Cuando la zona turística seleccionada no es la actual.}
			
			\UCpaso Inhabilita la opción \textbf{Generar ruta a partir de la ubicación actual}.
			
			\UCpaso Muestra la pantalla \IUref{LR-IU2.1a}{Zona turística}.
			
			\UCpaso [\UCactor] Solicita visualizar las actividades turísticas seleccionando la opción \textbf{Servicios turísticos cerca de mi}.
			
			\UCpaso Obtiene los servicios turísticos asociados a la zona turística previamente seleccionada.
			
			\UCpaso Continua en el paso \ref{LR-CU2.1:Pantalla} de la trayectoria principal.
		\end{UCtrayectoriaA}
	
	
	\begin{UCtrayectoriaA}[Fin del caso de uso]{B}{Cuando el turista requiere generar una ruta turística a partir de su ubicación actual.}
		
		\UCpaso [\UCactor] Solicita generar una ruta turística a partir de su ubicación actual seleccionando la opción \textbf{Generar ruta a partir de la ubicación actual} en la pantalla \IUref{LR-IU2.1a}{Zona turística}.
		
		\UCpaso Verifica que el turista se encuentre actualmente en una zona turística registrada. \refErr{Uno}
		
		\UCpaso \label{LR-CU2.1:EjecutaCaso} Ejecuta el caso de uso \cdtIdRef{LR-CU2.1.1}{Generar ruta turística}.
		
	\end{UCtrayectoriaA}
	
	
	\subsection{Puntos de extensión}
	\UCExtenssionPoint{El turista requiere visualizar la información de los servicios turísticos.}{En el paso \ref{LR-CU2.1:Pantalla} de la trayectoria principal.}{\cdtIdRef{LR-CU2.1.2}{Consultar información de servicios turísticos}.}
	
	\UCExtenssionPoint{El turista requiere generar una ruta turística a partir de su ubicación actual.}{En el paso \ref{LR-CU2.1:EjecutaCaso} de la trayectoria alterna \textbf{B}.}{\cdtIdRef{LR-CU2.1.1}{Generar ruta turística}.}