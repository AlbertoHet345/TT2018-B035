\vspace*{7cm}
\rightline{{\Huge \textcolor{sectionColor}{Justificación}}}
\vspace*{2cm}

\addcontentsline{toc}{chapter}{Justificación}

Ya que el uso del GPS ha ido trascendiendo en los últimos años se propone desarrollar una aplicación móvil en la cual se utilice el posicionamiento de una persona, en este caso un turista, y mediante el uso de geovallas se puedan trazar rutas turísticas, así como centros de interés dentro de esta geovalla o zona en la que se encuentre. \\

Nuestro proyecto aborda el e-tourism qué es el reflejo de la digitalización de todos los procesos y cadenas de valor de las industrias de turismo, viajes, hostelería y restauración. \\

El sector de la industria móvil se desarrolla en torno a un cliente que sitúa, en el núcleo de sus exigencias, el valor por el tiempo. Se ha estimado que cada clic adicional que un usuario de este tipo de dispositivos efectúa, reduce las posibilidades de que la transacción se realice un 50\%. Por tanto, se puede determinar que el comportamiento del usuario móvil, premia la accesibilidad y la velocidad \cite{mtourism}. \\

La parte medular de este trabajo estará en el uso de geovallas. Las cuales proporciona un servicio contextual cuando los usuarios ingresan o salen de un área de interés específica, dependiendo del lugar donde se ubiquen. \\

En ocasiones se requieren utilizar una aplicación en particular como en un aeropuerto o un centro comercial. Es así como las geovallas permiten definir perímetros, lo que acciona notificaciones o alertas cuando el dispositivo cruza un área delimitada.\\

Para el desarrollo del proyecto se pretende definir geovallas alrededor de zonas turísticas, permitiendo al usuario que ingrese, notificarle de los servicios que se ofrecen en la zona mostrando información descriptiva de estos \cite{mtourism}.\\

Una vez que la aplicación a través de la geovalla identifique que el usuario ingresó a una zona turística marcada, se podrá visualizar un listado de los servicios disponibles en el área o mostrar estos mismos servicios en el mapa para su mejor localización. Así mismo se generarán rutas opcionales para recorrer los destinos más importantes de la zona optimizando las distancias y los tiempos de recorridos.\\

Este proyecto se llevará a cabo en colaboración con estudiantes de la Escuela Superior de Turismo del Instituto Politécnico Nacional, elaborando un proyecto de investigación para el desarrollo del sistema planteado, el cual determine todo lo relacionado con las zonas turísticas, los servicios que se ofrecen y la información que los  describe. \\

Cabe resaltar que la delimitación del proyecto queda sujeto a lo establecido anteriormente y no contempla conexión con servicios de hostelería y servicios de restaurantes, así como ningún servicio turístico externo, excluyendo cualquier tipo de pago o reservación con dichos servicios. \\
