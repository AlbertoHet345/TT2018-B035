\vspace*{7cm}
\rightline{{\Huge \textcolor{sectionColor}{Justificación}}}
\vspace*{2cm}

\addcontentsline{toc}{chapter}{Justificación}

Ya que el uso del GPS ha ido trascendiendo en los últimos años, se propone desarrollar una aplicación móvil en la cual se utilice el posicionamiento de una persona, en este caso un turista, y mediante el uso de geocerca se puedan trazar rutas de recorrido, así como centros de interés dentro de esta geocerca o zona en la que se encuentre. \\

El proyecto aborda el \textit{e-tourism}\footnote{Turismo electrónico} que es el reflejo de la digitalización de todos los procesos y cadenas de valor de las industrias de turismo, viajes, hostelería y restauración. \\

El sector de la industria móvil se desarrolla en torno a un cliente que se caracteriza por dar prioridad al tiempo por sobre otros factores. Se ha estimado que cada clic adicional que un usuario de este tipo de dispositivos efectúa, reduce las posibilidades de que la transacción se realice un 50\%. Por tanto, se puede determinar que el comportamiento del usuario móvil, premia la accesibilidad y la velocidad \cite{mtourism}. \\

La parte medular de este trabajo está en el uso de geocercas, las cuales proporcionan un servicio contextual cuando los usuarios ingresan o salen de esta área de interés específica, dependiendo del lugar donde se ubiquen. \\

Actualmente hay aplicaciones particulares que registran el momento en que el usuario se posiciona en un aeropuerto, centro comercial o sitio de interés; es así como las geocercas permiten definir perímetros, lo que acciona notificaciones o alertas cuando el dispositivo cruza un área delimitada.\\

Para el desarrollo del proyecto se pretende definir o trazar geocercas alrededor de zonas turísticas, permitiendo al usuario que al ingresar, se le notifique de los servicios que se ofrecen en la zona, mostrando información descriptiva \cite{mtourism}.\\

Una vez que la aplicación, a través de la geocerca, identifique que el usuario ingresó a una zona turística marcada; se podrá visualizar un listado de los servicios disponibles en el área o mostrar estos mismos lugares en el mapa para su mejor localización. Así mismo el usuario podrá generar rutas para recorrer los destinos más importantes de la zona, optimizando la distancia y el tiempo empleado en estos recorridos.\\

%Este proyecto se llevará a cabo en colaboración con estudiantes de la Escuela Superior de Turismo del Instituto Politécnico Nacional, elaborando un proyecto de investigación para el desarrollo del sistema planteado, el cual determine todo lo relacionado con las zonas turísticas, los servicios que se ofrecen y la información que los describe. \\

El proyecto no contempló conexión con servicios de hostelería y servicios de restaurantes, así como ningún servicio turístico externo, excluyendo cualquier tipo de pago o reservación con dichos servicios. \\
