\chapter{Conclusiones}

\section{Conclusiones generales}

El Trabajo Terminal, es la integración de los conocimientos adquiridos a lo largo de la trayectoria escolar de la Ingeniería en Sistemas Computacionales, gracias a ello adquirimos 
una formación sólida para llevar a cabo la administración, gestión, análisis, diseño, desarrollo e implementación de sistemas computacionales que serán capaces de solucionar problemas en la vida diaria.\\

A lo largo del programa académico se cursaron unidades de aprendizaje que fueron de gran apoyo para la realización el Trabajo Terminal, tanto documentación como el desarrollo del sistema.Sin embargo, se tuvo que realizar una búsqueda de información acerca de las zonas turísticas del país, así como los servicios turísticos que éstas ofrecen, así mismo se investigó acerca del posicionamiento geográfico y el uso de geocercas (ya que al principio no se contaba con un conocimiento sólido del tema). \\

Una vez que se llevó a cabo la investigación sobre las áreas turísticas se determinó el uso de la tecnología a utilizar para poder llevar a cabo el Trabajo Terminal, aunado a esto, se realizó una investigación sobre la viabilidad del proyecto. A continuación se comenzó con el proceso de análisis en el que se determinaron los requerimientos funcionales y no funcionales para el sistema, así como los casos de uso; de la misma manera se realizó el proceso de diseño para determinar una arquitectura de solución y la manera en la que el usuario final se comunicará con el sistema.\\

Finalmente, el desarrollo del sistema requirió de un gran compromiso por parte de los integrantes. Además si se desea continuar con la propuesta en un futuro se podrá tomar los componentes desarrollados ya que la mayoría de estos cuentan con una licencia abierta, lo cual no sería un impedimento retomar y agregar nuevas funcionalidades al sistema.


\section{Trabajo a futuro}

\begin{enumerate}
	\item Implementar un servicio de posicionamiento más preciso que con el que se cuenta actualmente.
	
	\item Implementar un modo cartográfico en el que las rutas trazadas no requieran de caminos ya establecidos, como lo son carreteras, calles, avenidas, pasos peatonales, etc.
	
	\item Poder gestionar completamente un viaje turístico a varias zonas turísticas, es decir, poder trazar rutas entre zonas turísticas y no nada mas servicios.
	
\end{enumerate}