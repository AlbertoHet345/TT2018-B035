\section{Análisis de Viabilidad}

\subsection{Económico}

En México, la actividad económica se encarga de aportar casi el doble del promedio que contribuyen las economías de la Organización para la Cooperación y Desarrollo Económico (OCDE). Así mismo aporta el 8.7 \% del PIB del país. \\

Por otro lado, la industria turística ha crecido a nivel global, incluso por arriba de la economía mundial, sin embargo para asegurar la competitividad, sostenibilidad e inclusividad, se requieren políticas. \cite{turismoEnMexico}\\ 

El turismo en México es una actividad económica que tiene una gran importancia porque, como ya se mencionó, aporta un porcentaje alto en el PIB del país. Dado esto, México recibe anualmente un amplio caudal de turistas que provienen de todo el mundo y que genera un alto número de empleos locales.\cite{importanciaTurismo} \\

Según el análisis en el entorno económico y la relevancia del turismo en México, se puede llegar a la conclusión de que el país actualmente esta atravesando un excelente momento para el turismo, lo que demuestra la solidez y las excelentes expectativas que se tienen sobre la economía mexicana en el sector turístico.\\

\subsection{Técnico}

Para el desarrollo del proyecto se ha decidido hacer uso del trazado de áreas de interés (geovallas o geocercas), esta es una delimitación geográfica virtual que se hace a través de un programa de rastreo. El uso de esta tecnología permite administrar zonas o lugares que se encuentran dentro del área trazada o delimitada. \cite{geovalla}. \\

Para la implementación del la geocerca existen algunas API's, las cuales pueden ser utilizadas para el desarrollo del proyecto, algunas de estas API's son: Navixy\footnote{\url{https://www.navixy.com/es/documentacion/guias-de-usuario/interfaz-web/monitoreo/herramientas-del-mapa/geocercas/}} y RedGPS\footnote{\url{https://www.redgps.com/blog-noticias/bienvenidas-las-geocercas-a-redgps-1}}. Por lo que la implementación de estas geocercas se puede realizar.

 \subsection{Legal}

De manera legal el proyecto es viable, ya que el usuario final es el que limita el uso de la geolocalización, es decir, no hay una ley en México que restrinja su uso. 
