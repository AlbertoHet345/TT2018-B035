\section{Herramientas utilizadas}
A continuación se describen las herramientas que fueron utilizadas para el desarrollo del proyecto, dichas herramientas fueron de gran ayuda ya que permiten implementar aplicaciones de manera más rápida y más sencilla. 

\subsection{Herramientas para análisis y diseño}

Para el análisis y diseño se utilizaron las siguientes herramientas:

\begin{itemize}
	\item \textbf{TexStudio}: es un ambiente integrado de escritura para documentos LaTeX\cite{tex}.
	
	\item \textbf{Balsamiq}: es una herramienta de diseño de interfaces de usuario\cite{balsamiq}.
	
	\item \textbf{Visual Paradigm}: es un proveedor para negocio y transformación de tecnologías de la información\cite{visual}.
\end{itemize}

\subsection{Herramientas del módulo móvil}

Para el desarrollo del módulo móvil se utilizaron las siguientes herramientas:

\begin{itemize}
	
	\item \textbf{Xcode 11}: es un Entorno de Desarrollo Integrado (Integrated Development Environment, IDE por sus siglas en inglés\footnote{De aquí en adelante se empleará IDE para referirse a Entorno de Desarrollo Integrado}) proporcionado por Apple y permite el desarrollo nativo para todas sus plataformas (iOS, iPadOS, macOS, tvOS y watchOS)\cite{xcode}.
	
	\item \textbf{Mapbox}: es una API que permite la integración de mapas a alguna aplicación, ya sea móvil o web\cite{mapbox}.
	
	\item \textbf{Cosmos}: es un control de interfaz de usuario para iOS implementado en Swift, permite realizar puntuaciones utilizando estrellas para ello\cite{Cosmos}.
	
	\item \textbf{Swift 5.1}: es un lenguaje de programación de proposito general construido utilizando patrones de diseño de software\cite{swift}.
	
	\item \textbf{SideMenu}: es una librería para iOS que permite la inclusión de menus laterales dentro de la aplicación\cite{sidemenu}.
	
\end{itemize}

\subsection{Herramientas del módulo web}

Para el desarrollo del módulo web se utilizaron las siguientes herramientas:

\begin{itemize}
	
	\item \textbf{Node.js}: Concebido como un entorno de ejecución de JavaScript orientado a eventos asíncronos, Node.js está diseñado para construir aplicaciones en red escalables, se pueden manejar muchas conexiones concurrentes. Por cada conexión el callback será ejecutado, sin embargo si no hay trabajo que hacer Node.js estará durmiendo\cite{node}.
	
	\item \textbf{Mapbox}: es una API que permite la integración de mapas a alguna aplicación, ya sea móvil o web\cite{mapbox}.
	
	\item \textbf{sails}: es un marco de aplicación web Model-View-Controller (MVC) desarrollado sobre el entorno Node.js, lanzado como software libre y de código abierto bajo la Licencia MIT. Está diseñado para facilitar la creación de API y aplicaciones web Node.js de grado empresarial personalizadas\cite{sails}.
	
	
	\item \textbf{phpMyAdmin}: phpMyAdmin es un programa de libre distribución en PHP, creado por una comunidad sin ánimo de lucro, que sólo trabaja en el proyecto por amor al arte. Es una herramienta muy completa que permite acceder a todas las funciones típicas de la base de datos MySQL a través de una interfaz web muy intuitiva\cite{phpMyAdmin}.
		
	\item \textbf{jQuery}: jQuery es una biblioteca de JavaScript rápida, pequeña y rica en funciones. Hace que cosas como el desplazamiento y la manipulación de documentos HTML, el manejo de eventos, la animación y Ajax sean mucho más simples con una API fácil de usar que funciona en una multitud de navegadores. Con una combinación de versatilidad y extensibilidad\cite{jQuery}.
	
	\item \textbf{PHP}: PHP (acrónimo recursivo de PHP: Hypertext Preprocessor) es un lenguaje de código abierto muy popular especialmente adecuado para el desarrollo web y que puede ser incrustado en HTML\cite{PHP}.
	
		
	\item \textbf{CSS3}: El nombre hojas de estilo en cascada viene del inglés Cascading Style Sheets, del que toma sus siglas. CSS es un lenguaje usado para definir la presentación de un documento estructurado escrito en HTML o XML (y por extensión en XHTML). El W3C(World Wide Web Consortium) es el encargado de formular la especificación de las hojas de estilo que servirán de estándar para los agentes de usuario o navegadores\cite{CSS3}.
	
	\item \textbf{HTML}: siglas en inglés de HyperText Markup Language (‘lenguaje de marcas de hipertexto’), hace referencia al lenguaje de marcado para la elaboración de páginas web. Es un estándar que sirve de referencia del software que conecta con la elaboración de páginas web en sus diferentes versiones, define una estructura básica y un código (denominado código HTML) para la definición de contenido de una página web, como texto, imágenes, videos, juegos, entre otros\cite{HTML}.
	
\end{itemize}