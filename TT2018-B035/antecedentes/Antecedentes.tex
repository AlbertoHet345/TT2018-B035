\chapter{Antecedentes}

\section{Tendencias en Turismo}

Durante décadas, el turismo ha experimentado un continuo crecimiento y una profunda diversificación, hasta convertirse en uno de los sectores económicos que crecen con mayor rapidez en el mundo. El turismo mundial guarda una estrecha relación con el desarrollo y se inscriben en él un número creciente de nuevos destinos. Esta dinámica ha convertido al turismo en un motor clave del progreso socio-económico. \\

Este aspecto dinámico y expansivo del turismo se ha visto acompañado de una fuerza innovadora y creadora, con la que las ofertas intentan adecuarse cada vez más a las necesidades y a los deseos de las personas. Hoy el turismo presenta una gran variedad de formas y constituye una realidad plural y en continuo cambio.\\

Hoy en día, el volumen de negocio del turismo iguala o incluso supera al de las exportaciones de petróleo, productos alimentarios o automóviles. El turismo se ha convertido en uno de los principales actores del comercio internacional, y representa al mismo tiempo una de las principales fuentes de ingresos de numerosos países en desarrollo. Este crecimiento va de la mano del aumento de la diversificación y de la competencia entre los destinos.\\

La expansión general del turismo en los países industrializados y desarrollados ha sido beneficiosa, en términos económicos y de empleo, para muchos sectores relacionados, desde la construcción hasta la agricultura o las telecomunicaciones.

\section{Sistemas Existentes}

Las aplicaciones que se enumeran a continuación describen las similitudes y diferencias con la propuesta de este trabajo:

\begin{enumerate}
	\item \textbf{Skyscanner}. Es una aplicación Web y Móvil (iOS y Android), la cual verifica precios con al rededor de 1200 agencias de viajes, permite reservación de vuelo, renta de autos y alojamiento; guarda información de los viajes del usuario sin utilizarlos para aumento de precios, publicidad engañosa, etc. Cuenta con servicio al cliente disponible los siete días de la semana y ofrece soporte en 29 idiomas \cite{skyscanner}.  
	
	\item \textbf{Viator}. Cuenta con un acceso directo a más de 200000 actividades que se encuentran en más de 150 países. Permite reservar experiencias planificando salidas o sobre la marcha evitando filas o quedarse sin lugar. Es una aplicación Móvil (iOS y Android) y Web. De igual forma, cuenta con atención al cliente las 24 horas del día y ofrece una garantía de precios bajos y reembolso en caso de cancelación \cite{viator}. 
	
	\item \textbf{Triposo}. Permite elegir los hoteles favoritos, vistas, actividades y restaurantes, así mismo permite la descarga de guías para acceder a mapas, consejos, reservaciones y sugerencias personales; sin el uso de internet. Esta desarrollada para iOS y Android y tiene su API abierta para ser utilizada \cite{triposo}. 
	
	\item \textbf{TripAdvisor}. Una de las aplicaciones más populares de esta comparativa y tal vez la más conocida de las aplicaciones para buscar información sobre hoteles, vuelos, restaurantes y atractivos turísticos de todo el mundo. Su fuerte radica en las más de 75 millones de críticas y opiniones de viajeros que, junto a las fotos que suben, permiten conocer los detalles reales de cada sitio antes de ir o de hacer una reservación. Es posible acceder a opiniones, mapas y fotos para más de 300 ciudades de manera offline. Y, al estar conectado, la opción “Cerca de mí” muestra los hoteles, restaurantes y atractivos ubicados alrededor del viajero. Desarrollada para Android, IOS y Windows \cite{tripAdvisor}.
	
	\item \textbf{FourSquare}. El objetivo de esta aplicación es ayudar al usuario a encontrar restaurantes, bares, pubs y negocios en una infinidad de ciudades adaptándose a las selecciones y calificaciones realizadas con anterioridad por él mismo y por sus amigos. De este modo, se adapta a sus gustos y recomienda alternativas similares. Al funcionar con la geolocalización del usuario, esta herramienta es muy útil para saber qué lugares están cerca y obtener opiniones de la comunidad. Además, permite hacer búsquedas muy específicas, como un plato en un restaurante particular, o generales, por ejemplo, para encontrar los mejores bares con terraza. Al llegar a un sitio, Forsquare propone ingresar a la aplicación y realizar check in –como si se tratara de un hotel- para obtener tips privilegiados y ascender en sus niveles de expertise. Desarrollada para Android, IOS y Windows.\cite{FourSquare}
	
	\item  \textbf{Yelp}. Se basa, como Foursquare, en la geolocalización del usuario permitiéndole encontrar restaurantes, bares y negocios a su alrededor, filtrándolos por precio, distancia y los que están abiertos en ese momento. Cuenta con una gran comunidad activa que proporciona reseñas con fotos y valoraciones. El punto más destacado es que si el viajero hace una búsqueda y realiza un paneo con la cámara de fotos del teléfono o de la tablet Yelp muestra, como si fuera una brújula, los sitios ubicados a su alrededor. El usuario puede realizar check in al ingresar a alguno de los negocios para obtener descuentos y comenzar a obtener insignias cuando realiza,  por ejemplo, visitas repetidas a determinadas categorías de locales. Además, permite hacer reservaciones a través de OpenTable (sin tener que salir de la aplicación) y ofrece las direcciones y números telefónicos de miles de negocios.\cite{FourSquare}
	
\end{enumerate}

La principal diferencia que nuestro proyecto propone y que actualmente en el mercado no se maneja, es precisamente el guía turístico que con la geolocalización y las geovallas delimitan un área específica, que delimita la zona en donde la aplicación mostrará recomendaciones de lugares por visitar y servicios turísticos ofrecidos, generando una ruta a seguir para visitar de manera optima los sitios mas populares y tener un contexto informativo de dichos lugares.\\

Sin duda una de las características a tener en cuenta, es la cantidad de usuarios dentro de las plataformas, esto debido a que mientras mas usuarios activos tenga la aplicación, mayor será la cantidad de comentarios, recomendaciones y puntuaciones a los lugares recomendados por cualquier aplicación de guía turística, es por ello que nuestra propuesta abarcará, esta funcionalidad, para que los usuarios interactúen de manera directa en las opiniones y recomendaciones de los servicios turísticos, aun que cabe destacar que las zonas turísticas y las rutas se generarán en base a la recomendación de expertos en el área y la ubicación actual del usuario. \\

\section{El fenómeno del turismo}

El turísmo destaca por ser una de las actividades económicas más dinámicas y con mayor potencial de crecimiento a nivel mundial, esto debido a que esta muy relacionado con el ocio; que además, no sólo se han cambiado las prácticas turísticas, sino también la filosofía empresarial de la gestión turística. Así hoy en día podemos definir al turismo en un fenomeno integrado por dos partes: como una experiencia y una industria. Se debe tener en cuenta que el turismo es, ante todo, una industria social en la que se compran y venden experiencias. Estas experiencias, que están estrechamente ligadas al ocio, resultan la clave del éxito de la industria turística, además es importante destacar que es una de las actividades humanas más globalizadas y por ello ha cambiado y evolucionado a través del tiempo; ya que se transforma mediante la creación de nuevas tendencias así como nuevos destinos turísticos.\cite{fenom}\\

Así mismo, en la actualidad el turista ha cambiado, se informa antes de emprender su viajes de las ofertas y destinos que más le interesan, así como de la calidad de los servicios que demanda, ahora el turista es un conocedor, y se apoya principalmente de los servicios proporcionados en Internet para elegir el destino, teniendo siempre presente las vivencias, que pueda ofrecerle, además de la infraestructura de comunicaciones, que le permitan acceder de manera cómoda y segura. Por lo anterior es que la competencia a nivel internacional se ha multiplicado y diversificado, no solo por la evolución natural de los países, sino por la incorporación de nuevos y variados servicios turístico, como respuesta a la demanda del mercado, de tal manera que los conceptos de hospedaje evolucionaron con el propósito de retenerlo el mayor tiempo posible, y satisfacer todas las necesidades del viajero.\cite{fenom}
Sistemas Existentes 

Las aplicaciones descritas a continuación describen las similitudes y diferencias con la propuesta de este trabajo:
1. Trip by Skyscanner. Anteriormente conocida como Gogobot. Proporciona recomendaciones personalizadas sobre restaurantes, alojamientos y qué ver y hacer en cada lugar. Gogobot permite crear itinerarios, armar categorías y reordenar la lista de sitios a visitar. Al registrarse e ingresar los intereses del usuario, la aplicación comienza a brindar recomendaciones de acuerdo a gustos personales. La comunidad de Trip aporta sugerencias organizadas de acuerdo al tipo de viajero: aventureros, amantes de la noche, viajes de lujo o con bajo presupuesto, entre muchos otros. Permite aprovechar la experiencia de otros usuarios experimentados, conectarse con quienes tienen intereses similares y seguir a los más destacados. Además, muestra el pronóstico del tiempo de los siguientes 5 días de la ciudad donde se encuentra el usuario o del destino del viaje y permite enviar imágenes como postales con las aventuras vividas. Desarrollada para Android y IOS. En resumen esta app te ayuda a encontrar los mejores planes, restaurantes y hoteles basándose en tus intereses, el clima, la hora del día y tu ubicación.

2. Viator. Es una aplicación de experiencias turísticas. En su interfaz se encuentran y se pueden reservar más de 9000 tours y actividades en más de 150 países. Su fuerte reside en el amplio abanico de paseos que presenta organizado en distintos tipos de tours. Los tours se organizan por rubro, como shows y conciertos, tickets, excursiones y viajes en el día, en familia, a pie o en bicicleta, entre otros. Cada itinerario cuenta con una descripción detallada, precios, fechas, duración, lugar de encuentro, idioma del tour y opiniones de quienes que realizaron el mismo paseo. Pueden pagarse con tarjeta de crédito. Para quienes se registren, cuentan con descuentos en tours en algunas ciudades. Desarrollada para Android y IOS. Esta aplicación pertenece a la empresa de TripAdvisor.

4. Triposo. Esta aplicación de viajes que ofrece mapas de las ciudades más importantes, información resumida de cada destino e itinerarios de modo offline de todos los países del mundo. Cuenta también con un convertidor de moneda, con el pronóstico del tiempo, presenta las especialidades gastronómicas del lugar y un libro de frases útiles traducidas en el idioma del país que se va a visitar. Además, incluye guías de restaurantes, sugerencias de bares, vida nocturna y permite reservar hoteles, tickets y tours turísticos. Triposo considera la ubicación, hora y clima del lugar donde se encuentra el viajero para hacerle recomendaciones. Desarrollada para Android y IOS.

5. TripAdvisor. Una de las aplicaciones más populares de esta comparativa y tal vez la más conocida de las aplicaciones para buscar información sobre hoteles, vuelos, restaurantes y atractivos turísticos de todo el mundo. Su fuerte radica en las más de 75 millones de críticas y opiniones de viajeros que, junto a las fotos que suben, permiten conocer los detalles reales de cada sitio antes de ir o de hacer una reservación. Es posible acceder a opiniones, mapas y fotos para más de 300 ciudades de manera offline. Y, al estar conectado, la opción “Cerca de mí” muestra los hoteles, restaurantes y atractivos ubicados alrededor del viajero. Desarrollada para Android, IOS y Windows.
6. FourSquare. El objetivo de esta aplicación es ayudar al usuario a encontrar restaurantes, bares, pubs y negocios en una infinidad de ciudades adaptándose a las selecciones y calificaciones realizadas con anterioridad por él mismo y por sus amigos. De este modo, se adapta a sus gustos y recomienda alternativas similares. Al funcionar con la geolocalización del usuario, esta herramienta es muy útil para saber qué lugares están cerca y obtener opiniones de la comunidad. Además, permite hacer búsquedas muy específicas, como un plato en un restaurante particular, o generales, por ejemplo, para encontrar los mejores bares con terraza. Al llegar a un sitio, Forsquare propone ingresar a la aplicación y realizar check in –como si se tratara de un hotel- para obtener tips privilegiados y ascender en sus niveles de expertise. Desarrollada para Android, IOS y Windows.
7. Yelp. Se basa, como Foursquare, en la geolocalización del usuario permitiéndole encontrar restaurantes, bares y negocios a su alrededor, filtrándolos por precio, distancia y los que están abiertos en ese momento. Cuenta con una gran comunidad activa que proporciona reseñas con fotos y valoraciones. El punto más destacado es que si el viajero hace una búsqueda y realiza un paneo con la cámara de fotos del teléfono o de la tablet Yelp muestra, como si fuera una brújula, los sitios ubicados a su alrededor. El usuario puede realizar check in al ingresar a alguno de los negocios para obtener descuentos y comenzar a obtener insignias cuando realiza,  por ejemplo, visitas repetidas a determinadas categorías de locales. Además, permite hacer reservaciones a través de OpenTable (sin tener que salir de la aplicación) y ofrece las direcciones y números telefónicos de miles de negocios.
La principal diferencia que nuestro proyecto propone y que actualmente en el mercado no se maneja, es precisamente el guía turístico que con la geolocalización y las geovallas delimitan un área específica, que delimita la zona en donde la aplicación mostrará recomendaciones de lugares por visitar y servicios turísticos ofrecidos, generando una ruta a seguir para visitar de manera optima los sitios mas populares y tener un contexto informativo de dichos lugares.
Sin duda una de las características a tener en cuenta, es la cantidad de usuarios dentro de las plataformas, esto debido a que mientras mas usuarios activos tenga la aplicación, mayor será la cantidad de comentarios, recomendaciones y puntuaciones a los lugares recomendados por cualquier aplicación de guía turística, es por ello que nuestra propuesta abarcará, esta funcionalidad, para que los usuarios interactúen de manera directa en las opiniones y recomendaciones de los servicios turísticos, aun que cabe destacar que las zonas turísticas y las rutas se generarán en base a la recomendación de expertos en el área y la ubicación actual del usuario.




El fenómeno del turismo
Cabe resaltar que además de ser una de las actividades económicas más dinámicas y con mayor potencial de crecimiento a nivel mundial, el turismo está estrechamente ligado al fenómeno del ocio; que además, no sólo se han cambiado las prácticas turísticas, sino también la filosofía empresarial de la gestión turística. De este modo, el turismo se presenta en la actualidad como un fenómeno dual: como una experiencia y una industria. No podemos olvidar que el turismo es, ante todo, una industria social en la que se compran y venden experiencias. Estas experiencias, que están estrechamente ligadas al ocio, resultan la clave del éxito de la industria turística, sin pasar por alto que es una de las actividades humanas más globalizadas y por ello ha cambiado y evolucionado a través del tiempo; ya que se transforma mediante la creación de nuevas tendencias así como nuevos destinos turísticos.
De igual manera, en la actualidad el turista se ha transformado, se informa antes de emprender su viajes de las ofertas de destinos que más le interesan, así como de la calidad de los servicios que demanda, ahora el turista es un experto, y se apoya principalmente en la internet para elegir el destino, teniendo siempre presente las vivencias, que pueda ofrecerle, además de la infraestructura de comunicaciones, que le permitan acceder de manera cómoda y segura. Por lo anterior es que la competencia a nivel internacional se ha multiplicado y diversificado, no solo por la evolución natural de los países, sino por la incorporación de nuevos y variados servicios turístico, como respuesta a la demanda del mercado, de tal manera que los conceptos de hospedaje evolucionaron con el propósito de retenerlo el mayor tiempo posible, y satisfacer todas las necesidades del viajero.

\section{Servicios Turísticos}

\section{Problemática}

El uso de dispositivos electrónicos en la actualidad representa una sociedad más comunicada, por lo que el acceso a la información ha aumentado significativamente en los últimos años, esto implica que el uso de tecnología en tareas cotidianas vaya en aumento.\\

Las zonas turísticas en la actualidad se encuentran delimitadas en un área específica pero con el crecimiento acelerado del turismo, hoy en día la afluencia a estas zonas es cada vez mayor, lo que representa también una mayor expansión de los servicios turísticos, dando como resultado una cantidad amplia de alternativas para los visitantes.\\

En muchas ocasiones debido a la alta demanda de visitantes, las zonas turísticas se encuentran a su máxima capacidad, impidiendo concretar a los turistas el recorrido deseado, aunado a esto, y por cuestiones de dimensiones, es difícil para el viajero conocer los aspectos más importantes de una determinada zona; es por ello la necesidad de contar con guías turísticos, quienes dotan de información al visitante para planear de manera efectiva su ruta y así mismo las actividades a realizar.\\

Desde los horarios, actividades, eventos y hasta servicios en general es información fundamental para cualquier turista y ya que no siempre existe la facilidad de los módulos de atención turística gratuita que las entidades federativas proporcionan el contratar guías turísticos privados resultan un gasto económico por separado.

\section{Propuesta de Solución}