\section{Tendencias en Turismo}

Por décadas, el sector turístico ha sufrido un crecimiento continuo y una gran diversificación, en la actualidad se ha convertido en uno de los sectores económicos que ha crecido con mayor rapidez en el mundo. En el turismo mundial se encuentra en una gran relaciñon con el desarrollo y con un número de destinos, cada vez más creciente. Gracias a esto, el turismo se ha convertido en una pieza fundamental del progreso socio-económico \cite{turismo}.\\

Dado a la innata disposición a desplazarse del hombre, el turismo llama la atención debido a las dimensiones que ha alcanzado y sus perspectivas que tiene de expanción. Gracias a este aspecto, el turismo se ha visto acompañado de innovación y creación, permitiendo que las ofertas puedan adecuarse a las necesidades y deseos de las personas. De igual manera el turismo, hoy en día, presenta una gran variedad de formas y se encuentra en un continuo cambio \cite{realidadTurismo}. \\

En la actualidad, el negocio del turismo ha crecido igualando, incluso superando a los negocios de producción de automóviles, producción de productos alimenticios o a la exportación  de petróleo. Debido a esto, el turismo se ha convertido en una de la fuentes principales del comercio internacional, y es de las principales fuentes de ingresos para numerosos países en desarrollo.\\

Para los países desarrollados e industrializados ha resultado beneficiosa la expansión general del turismo, tanto en términos económicos, como en empleo.