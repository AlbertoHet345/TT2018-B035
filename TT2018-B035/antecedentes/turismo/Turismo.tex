\section{Tendencias en Turismo}

Durante décadas, el turismo ha experimentado un continuo crecimiento y una profunda diversificación, hasta convertirse en uno de los sectores económicos que crecen con mayor rapidez en el mundo. El turismo mundial guarda una estrecha relación con el desarrollo y se inscriben en él un número creciente de nuevos destinos. Esta dinámica ha convertido al turismo en un motor clave del progreso socio-económico. \\

Este aspecto dinámico y expansivo del turismo se ha visto acompañado de una fuerza innovadora y creadora, con la que las ofertas intentan adecuarse cada vez más a las necesidades y a los deseos de las personas. Hoy el turismo presenta una gran variedad de formas y constituye una realidad plural y en continuo cambio.\\

Hoy en día, el volumen de negocio del turismo iguala o incluso supera al de las exportaciones de petróleo, productos alimentarios o automóviles. El turismo se ha convertido en uno de los principales actores del comercio internacional, y representa al mismo tiempo una de las principales fuentes de ingresos de numerosos países en desarrollo. Este crecimiento va de la mano del aumento de la diversificación y de la competencia entre los destinos.\\

La expansión general del turismo en los países industrializados y desarrollados ha sido beneficiosa, en términos económicos y de empleo, para muchos sectores relacionados, desde la construcción hasta la agricultura o las telecomunicaciones.