\section{Servicios Turísticos}

Los Servicios Turísticos son el conjunto de  realizaciones, hechos y actividades, tendientes a producir prestaciones personales que satisfagan las necesidades del turista y contribuyan al logro de facilitación, acercamiento, uso y disfrute de los bienes turísticos. \\

Según la OEA (1980), los Servicios Turísticos, se describen como el resultado de las funciones, acciones y actividades que ejecutadas coordinadamente, por el sujeto receptor, permiten satisfacer al turista, hacer uso óptimo de las facilidades o industria turística y darle valor económico a los atractivos o recursos turísticos.\\

Tienen la consideración de servicios turísticos:

\begin{itemize}
	\item Servicio de alojamiento: cuando se facilite hospedaje o estancia a los usuarios de servicios turísticos, con o sin prestación de otros servicios complementarios.
	
	\item Servicio de alimentación: cuando se proporcione alimentos o bebidas para ser consumidas en el mismo establecimiento o en instalaciones ajenas.
	
	\item Servicio de guía: cuando se preste servicios de recorridos turísticos profesionales, para interpretar el patrimonio natural y cultural de un lugar.
	
	\item Servicio de OPC: cuando se brinde organización de eventos como reuniones, congresos, seminarios o convenciones.
	
	\item Servicio de información: cuando se facilite información a usuarios de servicios turísticos sobre recursos turísticos, con o sin prestación de otros servicios complementarios.
	
	\item Servicio de intermediación: Agencias de Viajes, cuando en la prestación de cualquier tipo de servicio turístico susceptible de ser demandado por un usuario, intervienen personas como medio para facilitarlos.
	
	\item Servicios de consultoría turística: dado por especialistas licenciados en el sector turismo para realizar la labor de consultoría turística.
	
	\item Servicios de transporte: ofrecido por la necesidad de los turistas a movilizarse.
	
\end{itemize}