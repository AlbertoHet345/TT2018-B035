\section{Propuesta de Solución}

Teniendo en mente la problemática descrita en el capítulo anterior, nos enfocaremos en describir las ventajas que resultan a partir de nuestra aplicación propuesta.\\ 

Primeramente, cabe resaltar el avance tecnológico en la industria móvil, hoy en día la cantidad de usuarios es tan grande que en términos generales el 84 porciento de los mexicanos posee algún dispositivo móvil y 4 de cada 10 utilizan un smartphone, por lo que nuestro producto va enfocado a un mercado amplio.\\ 

A su vez el usar dispositivos móviles no garantiza el uso de la amplia gama de aplicaciones que se desarrollan hoy en día por lo que las más premiadas son las de mayor accesibilidad y velocidad. Destacando esto, el e-turismo ha destacado en los últimos años como una herramienta fundamental para todo el sector turístico, desde aplicaciones de hostelería, restaurantes, alojamiento privado, trasporte, paquetes de viaje hasta comparadores que ayudan al viajero a encontrar las mejores opciones. Es por ello por lo que en un sector cada vez mas apoyado en la tecnología, decidimos incursionar con una aplicación competitiva que ofrezca la información de un asistente turístico.\\


Para innovar nos hemos dado a la tarea de plantear una asistente turístico que como principal característica sea la generación de rutas, las cuales resulten las más efectivas, acorde a las distancias y lugares de interés en una determinada zona, además que la aplicación pretende informar a través de notificaciones cuando el usuario ingresó a un área turística, todo esto sustentado en lo que se conoce como geovallas que delimitan una zona geográfica específica y con la ayuda de las tecnologías GPS cada vez más precisas se pretende que el turista tenga una experiencia de usuario eficiente. 
Aunado a nuestra característica principal, se plantea tener un catálogo de los sitios de interés mas relevantes de la zona, con la información de horarios, resumen cultural e histórico y fotografías. También se pretende contar con un catálogo similar asociado a los servicios turísticos de la zona como hostelería, restaurantes y actividades, todo esto con la capacidad de mostrar en el mapa su ubicación.\\

Para sustentar la información turística que se integrará a la aplicación se hará una colaboración con expertos en la rama los cuales llevaran a cabo un proyecto de investigación el cual determine todo lo relacionado con las zonas turísticas, los servicios que se ofrecen y la información que los describe.