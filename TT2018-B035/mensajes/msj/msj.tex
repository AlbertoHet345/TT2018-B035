\begin{mensaje}{MSG1}{Operación exitosa}{Notificación Emergente}
	\item[Objetivo:] Notificar al usuario que se realizó con exito su petición  
	\item[Redacción:] Muestra en pantalla el siguiente mensaje “operación exitosa pulse para continuar” en una ventana emergente con un único botón para aceptar.
	\item[Ejemplo:] operación exitosa pulse para continuar
\end{mensaje}

\begin{mensaje}{MSG2}{Correo inválido}{Notificación en Campo}
	\item[Objetivo:] Notificar al usuario que ingresó un correo incorrecto o nulo.  
	\item[Redacción:] Muestra en pantalla arriba del campo correo electrónico el texto “Correo inválido”.
	\item[Ejemplo:] Correo inválido
\end{mensaje}

\begin{mensaje}{MSG3}{Correo existente}{Notificación en Campo}
	\item[Objetivo:] Notificar al usuario que ingresó un correo previamente registrado.
	\item[Redacción:] Muestra en pantalla arriba del campo correo electrónico el texto “Correo registrado”.
	\item[Ejemplo:] Correo registrado.
\end{mensaje}

\begin{mensaje}{MSG4}{Formato de campo inválido}{Notificación en Campo}
	\item[Objetivo:] Notificar al usuario que ingresó un campo con un formato inválido.  
	\item[Redacción:] Muestra en pantalla arriba del campo con formato incorrecto el texto “Formato de campo inválido”.
	\item[Ejemplo:] Formato de campo inválido
\end{mensaje}

\begin{mensaje}{MSG5}{Campo Obligatorio}{Notificación en Campo}
	\item[Objetivo:] Notificar al usuario que necesita marcar o no dejar nulo un campo.
	\item[Redacción:] Muestra en pantalla arriba del campo nulo el texto “Campo Obligatorio”.
	\item[Ejemplo:] Campo Obligatorio
\end{mensaje}

\begin{mensaje}{MSG6}{Operación fallida}{Notificación Emergente}
	\item[Objetivo:] Notificar al usuario que no se pudo realizar su petición. 
	\item[Redacción:] Muestra en pantalla el siguiente mensaje “Operación denegada pulse para continuar” en una ventana emergente con un único botón para aceptar.
	\item[Ejemplo:] Operación denegada pulse para continuar
\end{mensaje}

\begin{mensaje}{MSG7}{Correo y/o contraseña incorrectos}{Notificación Emergente}
	\item[Objetivo:] Notificar al usuario que la credenciales ingresadas son incorrectas. 
	\item[Redacción:] Muestra en pantalla el siguiente mensaje “Correo y/o contraseña incorrectos pulse para continuar” en una ventana emergente con un único botón para aceptar.
	\item[Ejemplo:] Correo y/o contraseña incorrectos pulse para continuar
\end{mensaje}

\begin{mensaje}{MSG8}{Confirmación de eliminar historial}{Notificación Emergente}
	\item[Objetivo:] Mostrar al usuario la confirmación de que desea eliminar el historial o la posibilidad de cancelar la operación. 
	\item[Redacción:] Muestra en pantalla el siguiente mensaje “Esta seguro que dese eliminar el historial de sus visitas, esto no afectará los comentarios y publicaciones registradas” en una ventana emergente con los botones Sí y No para aceptar o rechazar la petición.
	\item[Ejemplo:] Esta seguro que dese eliminar el historial de sus visitas, esto no afectará los comentarios y publicaciones registradas.
\end{mensaje}
