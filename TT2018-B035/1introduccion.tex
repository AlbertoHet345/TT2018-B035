%=========================================================
\chapter{Introducción}

En la actualidad el uso de dispositivos móviles, tales como smartphones y tabletas electrónicas, ha incrementado. Estos cada vez se equipan con nuevos sensores para el monitoreo del entorno, uno de estos sensores es el GPS (Global Positioning System), el cual permite saber la localización del usuario. Gracias al GPS surgen los servicios basados en localización (LBS, por sus siglas en inglés).\\

El sistema de posicionamiento global proporciona a los usuarios información sobre posicionamiento, navegación y cronometría, el servicio es gratuito y está a disposición de todos los usuarios de manera permanente y global. Esto  ha permitido a los usuarios de todo el mundo desarrollar múltiples aplicaciones y nuevos usos del GPS.\\

En la actualidad los teléfonos inteligentes habilitados para GPS suelen tener una precisión de 4,9 m (16 pies) de radio en cielo abierto. Sin embargo, su exactitud empeora cerca de edificios, puentes y árboles.\\

Los usuarios de gama alta aumentan la precisión del GPS con receptores de doble frecuencia y / o sistemas de aumentación. Estos pueden permitir el posicionamiento en tiempo real dentro de unos pocos centímetros.\cite{gps}




\section{Turismo en Expansión}



Durante décadas, el turismo ha experimentado un continuo crecimiento y una profunda diversificación, hasta convertirse en uno de los sectores económicos que crecen con mayor rapidez en el mundo. El turismo mundial guarda una estrecha relación con el desarrollo y se inscriben en él un número creciente de nuevos destinos. Esta dinámica ha convertido al turismo en un motor clave del progreso socio-económico. \\

Este aspecto dinámico y expansivo del turismo se ha visto acompañado de una fuerza innovadora y creadora, con la que las ofertas intentan adecuarse cada vez más a las necesidades y a los deseos de las personas. Hoy el turismo presenta una gran variedad de formas y constituye una realidad plural y en continuo cambio.\\

Hoy en día, el volumen de negocio del turismo iguala o incluso supera al de las exportaciones de petróleo, productos alimentarios o automóviles. El turismo se ha convertido en uno de los principales actores del comercio internacional, y representa al mismo tiempo una de las principales fuentes de ingresos de numerosos países en desarrollo. Este crecimiento va de la mano del aumento de la diversificación y de la competencia entre los destinos.\\

La expansión general del turismo en los países industrializados y desarrollados ha sido beneficiosa, en términos económicos y de empleo, para muchos sectores relacionados, desde la construcción hasta la agricultura o las telecomunicaciones.





\section{Los Servicios Turísticos}



Los Servicios Turísticos son el conjunto de  realizaciones, hechos y actividades, tendientes a producir prestaciones personales que satisfagan las necesidades del turista y contribuyan al logro de facilitación, acercamiento, uso y disfrute de los bienes turísticos. \\

Según la OEA (1980), los Servicios Turísticos, se describen como el resultado de las funciones, acciones y actividades que ejecutadas coordinadamente, por el sujeto receptor, permiten satisfacer al turista, hacer uso óptimo de las facilidades o industria turística y darle valor económico a los atractivos o recursos turísticos.\\

Tienen la consideración de servicios turísticos:

	\begin{itemize}
	 	\item Servicio de alojamiento: cuando se facilite hospedaje o estancia a los usuarios de servicios turísticos, con o sin prestación de otros servicios complementarios.
	
	 	\item Servicio de alimentación: cuando se proporcione alimentos o bebidas para ser consumidas en el mismo establecimiento o en instalaciones ajenas.
	
	 	\item Servicio de guía: cuando se preste servicios de recorridos turísticos profesionales, para interpretar el patrimonio natural y cultural de un lugar.
	
		\item Servicio de OPC: cuando se brinde organización de eventos como reuniones, congresos, seminarios o convenciones.
	
	 	\item Servicio de información: cuando se facilite información a usuarios de servicios turísticos sobre recursos turísticos, con o sin prestación de otros servicios complementarios.
	
		\item Servicio de intermediación: Agencias de Viajes, cuando en la prestación de cualquier tipo de servicio turístico susceptible de ser demandado por un usuario, intervienen personas como medio para facilitarlos.
	
		\item Servicios de consultoría turística: dado por especialistas licenciados en el sector turismo para realizar la labor de consultoría turística.
	
	 	\item Servicios de transporte: ofrecido por la necesidad de los turistas a movilizarse.
	
	\end{itemize}



\section{Problemática}

	
	El uso de dispositivos electrónicos en la actualidad representa una sociedad más comunicada, por lo que el acceso a la información ha aumentado significativamente en los últimos años, esto implica que el uso de tecnología en tareas cotidianas vaya en aumento.\\
	
	Las zonas turísticas en la actualidad se encuentran delimitadas en un área específica y con el crecimiento acelerado del turismo hay en día la afluencia a estas zonas es cada vez mayor, lo que representa también una mayor expansión de los servicios turísticos, dando como resultado una cantidad amplia de alternativas para los visitantes.\\
	
	En muchas ocasiones debido a la alta demanda de visitantes las zonas turísticas se encuentran a su máxima capacidad impidiendo concretar a los turista el recorrido deseado, aunado a esto también por cuestiones de dimensiones es difícil para el viajero conocer los aspectos más importantes de una determinada zona, es ahí donde normalmente los guías turísticos, los cuales son expertos en el área, dotan de información al visitante para planear de manera efectiva su ruta y así mismo planear las actividades a realizar.\\
	
	Desde los horarios, actividades, eventos y hasta servicios en general es información fundamental para cualquier turista y ya que no siempre existe la facilidad de los módulos de atención turística gratuita que las entidades federativas proporcionan el contratar guías turísticos privados resultan un gasto económico por separado.


\section{Propuesta de Solución}


	Teniendo en mente la problemática descrita en el capítulo anterior, nos enfocaremos en describir las ventajas que resultan a partir de nuestra aplicación propuesta.\\ 
	
	Primeramente, cabe resaltar el avance tecnológico en la industria móvil, hoy en día la cantidad de usuarios es tan grande que en términos generales el 84 porciento de los mexicanos posee algún dispositivo móvil y 4 de cada 10 utilizan un smartphone, por lo que nuestro producto va enfocado a un mercado amplio.\\ 
	
	A su vez el usar dispositivos móviles no garantiza el uso de la amplia gama de aplicaciones que se desarrollan hoy en día por lo que las más premiadas son las de mayor accesibilidad y velocidad. Destacando esto, el e-turismo ha destacado en los últimos años como una herramienta fundamental para todo el sector turístico, desde aplicaciones de hostelería, restaurantes, alojamiento privado, trasporte, paquetes de viaje hasta comparadores que ayudan al viajero a encontrar las mejores opciones. Es por ello por lo que en un sector cada vez mas apoyado en la tecnología, decidimos incursionar con una aplicación competitiva que ofrezca la información de un asistente turístico.\\
	
	
	Para innovar nos hemos dado a la tarea de plantear una asistente turístico que como principal característica sea la generación de rutas, las cuales resulten las más efectivas, acorde a las distancias y lugares de interés en una determinada zona, además que la aplicación pretende informar a través de notificaciones cuando el usuario ingresó a un área turística, todo esto sustentado en lo que se conoce como geovallas que delimitan una zona geográfica específica y con la ayuda de las tecnologías GPS cada vez más precisas se pretende que el turista tenga una experiencia de usuario eficiente. 
	Aunado a nuestra característica principal, se plantea tener un catálogo de los sitios de interés mas relevantes de la zona, con la información de horarios, resumen cultural e histórico y fotografías. También se pretende contar con un catálogo similar asociado a los servicios turísticos de la zona como hostelería, restaurantes y actividades, todo esto con la capacidad de mostrar en el mapa su ubicación.\\
	
	Para sustentar la información turística que se integrará a la aplicación se hará una colaboración con expertos en la rama los cuales llevaran a cabo un proyecto de investigación el cual determine todo lo relacionado con las zonas turísticas, los servicios que se ofrecen y la información que los describe.

%\cdtInstrucciones{
%	Presentar el documento, indicando su contenido, a quien va dirigido, quien lo realizó, por que razón, dónde y cuando. \\
%}
%	Este documento contiene la Especificacion del ptoyecto ``{\em Nombre del proyecto}'' correspondiente al trabajo realizado en el semestre 2016-2017-2 para la materia de Análisis y diseño orientado a objetos en el grupo 2CV9 por el equipo {\em Nombre del equipo}.
%
%%---------------------------------------------------------
%\section{Presentación}
%
%
%\cdtInstrucciones{
%	Indique el propósito del documento y las distintas formas en que puede ser utilizado.\\
%}
%	Este documento contiene la especificación de los requerimientos del usuario y del sistema del sistema a desarrollar. Tiene como objetivo establecer la naturaleza y funciones del sistema para su evaluación al final del semestre. Este documento debe ser aprobado por los principales responsables del proyecto.
%	
%	Este documento es el C2-EP1 del proyecto ``{\em Nombre del proyecto}''.
%	
%%---------------------------------------------------------
%\section{Organización del contenido}
%
%	En el capítulo \ref{cap:reqUsr} ...
%	
%	En el capítulo \ref{cap:reqSist} ...
%
%%---------------------------------------------------------
%\section{Notación, símbolos y convenciones utilizadas}
%
%	Los requerimientos funcionales utilizan una clave RFX, donde:
%	
%\begin{description}
%	\item[X] Es un número consecutivo: 1, 2, 3, ...
%	\item[RF] Es la clave para todos los {\bf R}equerimientos {\bf F}uncionales.
%\end{description}
%
%	Los requerimientos del usuario utilizan una clave RUX, donde:
%	
%\begin{description}
%	\item[X] Es un número consecutivo: 1, 2, 3, ...
%	\item[RU] Es la clave para todos los {\bf R}equerimientos del {\bf U}suario.
%\end{description}
%
%	Además, para los requerimeitnos funcionales se usan las abreviaciones que se muestran en la tabla~\ref{tbl:leyendaRF}.
%\begin{table}[hbtp!]
%	\begin{center}
%    \begin{tabular}{|r l|}
%	    \hline
%    	{\footnotesize Id} & {\footnotesize\em Identificador del requerimiento.}\\
%    	{\footnotesize Pri.} & {\footnotesize\em Prioridad}\\
%    	{\footnotesize Ref.} & {\footnotesize\em Referencia a los Requerimientos de usuario.}\\
%    	{\footnotesize MA} & {\footnotesize\em Prioridad Muy Alta.}\\
%    	{\footnotesize A} & {\footnotesize\em Prioridad Alta.}\\
%    	{\footnotesize M} & {\footnotesize\em Prioridad Media.}\\
%    	{\footnotesize B} & {\footnotesize\em Prioridad Baja.}\\
%    	{\footnotesize MB} & {\footnotesize\em Prioridad Muy Baja.}\\
%		\hline
%    \end{tabular} 
%    \caption{Leyenda para los requerimientos funcionales.}
%    \label{tbl:leyendaRF}
%	\end{center}
%\end{table}