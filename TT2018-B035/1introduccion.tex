%=========================================================
\vspace*{7cm}
\rightline{{\Huge \textcolor{sectionColor}{Introducción}}}
\vspace*{2cm}

\addcontentsline{toc}{chapter}{Introducción}

En la actualidad el uso de dispositivos móviles, tales como teléfonos inteligentes (smartphones) y tabletas electrónicas, ha incrementado. Estos cada vez se equipan con nuevos sensores para el monitoreo del entorno, uno de estos sensores es el Sistema de Posicionamiento Global (Global Positioning System, GPS por sus siglas en inglés\footnote{De aquí en adelante se empleará GPS para referirse al Sistema de Posicionamiento Global}), el cual permite saber la localización del usuario. Gracias al GPS surgen los Servicios Basados en Localización (Local Based Services, LBS, por sus siglas en inglés\footnote{De aquí en adelante se empleará LBS para referirse a los Servicios Basados en Localización}).\\

El sistema de posicionamiento global proporciona a los usuarios información no solo de posicionamiento, sino de navegación y cronometría; el servicio es gratuito y está a disposición de todos los usuarios de manera permanente y global. Esto  ha permitido a los usuarios de todo el mundo desarrollar múltiples aplicaciones y nuevos usos del GPS.\\

Los teléfonos inteligentes habilitados para GPS suelen tener una precisión de 4.9 m (16 pies) de radio en cielo abierto, sin embargo, su exactitud se reduce cerca de edificios, puentes y árboles.\\

Los usuarios de gama alta experimentan un aumento en la precisión del GPS con receptores de doble frecuencia y/o sistemas de aumentación. Estos pueden permitir el posicionamiento en tiempo real dentro de unos pocos centímetros \cite{gps}. \\

En el presente documento se describen  los antecedentes, la problemática, así como la propuesta de solución y el análisis realizado para el desarrollo de dicha propuesta.\\

Para los antecedentes se realizó una investigación documental referida a los sistemas existentes, el impacto que tiene el turismo en México y la manera en que puede ser utilizado el proyecto que aquí se presenta para seguir impulsando el turismo en el país. \\

En la propuesta de solución se enfatiza la forma en que se aborda el problema descrito en los antecedentes, así como se muestra el análisis que se ha llevado a cabo durante el desarrollo del proyecto. \\

Posteriormente se hace un análisis de la solución que incluye los requerimientos funcionales y los no funcionales con los que debe contar el sistema para cumplir con los objetivos propuestos en el proyecto. Así mismo se hace el análisis de los casos de uso identificados para el sistema. \\

Finalmente el modelo de interacción es la etapa de diseño que se hace para satisfacer las necesidades del proyecto y de los casos de uso que fueron identificados. En esta etapa se lleva a cabo el diseño de interfaces de usuario y mensajes con los que debe interactuar el usuario final.
%\cdtInstrucciones{
%	Presentar el documento, indicando su contenido, a quien va dirigido, quien lo realizó, por que razón, dónde y cuando. \\
%}
%	Este documento contiene la Especificacion del ptoyecto ``{\em Nombre del proyecto}'' correspondiente al trabajo realizado en el semestre 2016-2017-2 para la materia de Análisis y diseño orientado a objetos en el grupo 2CV9 por el equipo {\em Nombre del equipo}.
%
%%---------------------------------------------------------
%\section{Presentación}
%
%
%\cdtInstrucciones{
%	Indique el propósito del documento y las distintas formas en que puede ser utilizado.\\
%}
%	Este documento contiene la especificación de los requerimientos del usuario y del sistema del sistema a desarrollar. Tiene como objetivo establecer la naturaleza y funciones del sistema para su evaluación al final del semestre. Este documento debe ser aprobado por los principales responsables del proyecto.
%	
%	Este documento es el C2-EP1 del proyecto ``{\em Nombre del proyecto}''.
%	
%%---------------------------------------------------------
%\section{Organización del contenido}
%
%	En el capítulo \ref{cap:reqUsr} ...
%	
%	En el capítulo \ref{cap:reqSist} ...
%
%%---------------------------------------------------------
%\section{Notación, símbolos y convenciones utilizadas}
%
%	Los requerimientos funcionales utilizan una clave RFX, donde:
%	
%\begin{description}
%	\item[X] Es un número consecutivo: 1, 2, 3, ...
%	\item[RF] Es la clave para todos los {\bf R}equerimientos {\bf F}uncionales.
%\end{description}
%
%	Los requerimientos del usuario utilizan una clave RUX, donde:
%	
%\begin{description}
%	\item[X] Es un número consecutivo: 1, 2, 3, ...
%	\item[RU] Es la clave para todos los {\bf R}equerimientos del {\bf U}suario.
%\end{description}
%
%	Además, para los requerimeitnos funcionales se usan las abreviaciones que se muestran en la tabla~\ref{tbl:leyendaRF}.
%\begin{table}[hbtp!]
%	\begin{center}
%    \begin{tabular}{|r l|}
%	    \hline
%    	{\footnotesize Id} & {\footnotesize\em Identificador del requerimiento.}\\
%    	{\footnotesize Pri.} & {\footnotesize\em Prioridad}\\
%    	{\footnotesize Ref.} & {\footnotesize\em Referencia a los Requerimientos de usuario.}\\
%    	{\footnotesize MA} & {\footnotesize\em Prioridad Muy Alta.}\\
%    	{\footnotesize A} & {\footnotesize\em Prioridad Alta.}\\
%    	{\footnotesize M} & {\footnotesize\em Prioridad Media.}\\
%    	{\footnotesize B} & {\footnotesize\em Prioridad Baja.}\\
%    	{\footnotesize MB} & {\footnotesize\em Prioridad Muy Baja.}\\
%		\hline
%    \end{tabular} 
%    \caption{Leyenda para los requerimientos funcionales.}
%    \label{tbl:leyendaRF}
%	\end{center}
%\end{table}